\chapter{引言}
\section{问题介绍}

在当今数字图像处理和计算机视觉领域,图像去噪一直是一个至关重要的问题。在实际应用中,由于各种因素,例如传感器噪声、压缩、低光条件等,图像中常常存在噪声,这会影响到图像的质量和后续处理任务的准确性。如何从现实中实际获取带噪图像中还原得到原先清晰度更高的原始图像具有重大的应用价值和前景。
因此,开发高效准确的图像去噪算法对于提升图像质量以及下游任务的性能具有重要意义。
\subsection*{图像去噪的挑战}
传统的图像去噪方法通常基于滤波器和统计学方法,它们可以在某些场景下取得不错的效果。然而,在面对复杂的噪声分布或者噪声与信号之间相互重叠的情况下,传统方法的性能往往受到限制。特别是对于低信噪比(SNR)的图像,传统方法可能会产生过度平滑或者细节损失的问题。      

若利用传统统计学习方法来进行图像处理,则通常将二维图像当作一维向量进行处理。图像作为一类特殊的数据类型其分布通常可以被看作在高维向量空间中的一个低维流形的嵌入,即在图像中通常有较多常见特征反复出现,使得其有相比于一般向量空间中数据分布的特殊性质,若对其向量化则会失去一部分图像数据分布独有的特征。
\subsection*{引入深度学习}
近年来,随着深度学习技术的发展,基于深度学习的图像去噪方法逐渐受到关注。其中,Variational Autoencoder(VAE)和Diffusion Model (DM)等深度生成模型因其能够学习到图像的复杂概率分布而备受瞩目。这些模型不仅能够准确地去除噪声,还可以保留图像的细节和结构,从而在实际应用中取得更好的效果。
\section{选题背景及意义}

\subsection*{选题背景}
随着数字图像在日常生活和工业应用中的普及,图像质量的要求越来越高。然而,在现实世界中,图像往往会受到各种形式的干扰和噪声,例如传感器噪声、压缩引起的伪像、低光条件下的图像模糊等。这些噪声不仅降低了图像的质量,还可能影响后续的图像分析和处理任务的准确性和可靠性。\par 
传统的图像去噪方法通常基于数学模型和信号处理技术,例如中值滤波、高斯滤波等。然而,这些方法往往只能处理特定类型的噪声,对于复杂的噪声分布或者噪声与信号之间的相互干扰,传统方法的效果可能不尽如人意,具有较大的局限性。
\subsection*{选题意义}
提升图像质量: 有效的图像去噪算法可以帮助提升图像的质量,使其更加清晰和真实,提高观感体验。\par 
改善后续处理任务的效果: 在许多图像处理和分析任务中,图像质量对结果的影响至关重要。去噪后的图像更容易进行特征提取、对象检测、图像识别等后续处理任务,从而提高了这些任务的准确性和鲁棒性。\par 
推动深度学习在图像处理领域的应用: 结合VAE和Diffusion Model等深度生成模型进行图像去噪不仅可以提高去噪效果,还可以推动深度学习技术在图像处理领域的应用。通过将深度学习引入图像去噪领域,可以充分挖掘图像数据的潜在信息,实现更加智能和高效的图像处理算法。\par 
实践应用的需求: 在实际应用中,如医学影像、卫星图像、安防监控等领域,图像质量对于决策和诊断的重要性不言而喻。因此,开发高效准确的图像去噪算法对于满足实践应用的需求具有重要意义。\par 
综上所述,研究图像去噪算法不仅有助于提升图像质量和后续处理任务的效果,还有助于推动深度学习技术在图像处理领域的应用,满足实际应用的需求,具有重要的理论和实践意义。
\subsection*{本研究的目的}
本研究旨在探索如何结合VAE和Diffusion Model这两种先进的深度生成模型来解决图像去噪问题。通过结合VAE的自编码器结构和Diffusion Model的概率建模能力,我们希望实现对于不同类型噪声的鲁棒去除,并保持图像的真实性和细节。本研究的结果不仅可以提升图像去噪的效果,还可能对于其他相关领域,如图像增强和图像重建等,产生积极的影响。通过对图像去噪问题的深入研究,我们有望为实际应用提供更加鲁棒和可靠的解决方案,推动计算机视觉技术在实践中的进一步发展。
\section{论文结构安排}
首先,在第二章探讨了从传统的非条件生成模型,如变分自编码器(VAE),到当代先进的扩散模型在图像修复领域的发展历程。分析了VAE在图像处理中的早期应用,评估其在捕捉数据集复杂分布方面的能力及局限性。随着深度学习技术的发展,扩散模型,如Score-Based模型和Denoising Diffusion Probabilistic Models (DDPMs),已被证明在无监督学习框架下更有效地学习数据分布。此外,本章还着重评述了最新的stable diffusion模型,这是一种结合了扩散过程和深度学习的技术,展现出在图像修复任务中的高效性和创新性。通过这些模型的比较,我们可以观察到无监督学习算法在图像修复领域日益增强的能力。    

第三章详细介绍了本文的核心问题——图像修复的下游任务。本章从定义图像修复的具体场景和挑战开始,进一步细化本文旨在解决的问题,即在给定前向加噪算子信息的情况下,通过加噪后图像在目标数据集中生成原像满足后验关系,并明确了研究的主要目的:提高修复质量,缩短处理时间,以及减少所需的计算资源。


在第四章中,主要叙述了

在第五章中,我们深入探讨了实验的设计和参数设置,详细描述了在实现工程化图像去噪任务中采用的技术和工具。通过严谨的实验设计和详尽的参数记录,本章展现了实验的可重复性。接着,本文对不同的条件生成模型进行了性能比较,使用定量指标和定性分析相结合的方式,论证了所提出方法的优势。对比分析涵盖了生成图像的清晰度、色彩保真度和结构一致性等多个维度。   

最终章节提供了图像修复领域未来发展的预测和建议。讨论了潜在的技术突破,如通过深度学习进一步优化模型的结构,引入更先进的正则化技术,以及利用未来可能出现的计算资源,如量子计算。同时,本章还强调了在道德和法规框架内开展研究的重要性,以及研究成果的社会责任。 

% \section{引言的写法}

% 一篇学位论文的引言大致包含如下几个部分:
% 1、问题的提出;
% 2、选题背 景及意义;
% 3、文献综述;
% 4、研究方法;
% 5、论文结构安排。
% \begin{itemize}
%   \item 问题的提出:要清晰地阐述所要研究的问题“是什么”。
%     \footnote{选题时切记要有“问题意识”,不要选不是问题的问题来研究。}
%   \item 选题背景及意义:论述清楚为什么选择这个题目来研究,即阐述该研究对学科发展的贡献、对国计民生的理论与现实意义等。
%   \item 文献综述:对本研究主题范围内的文献进行详尽的综合述评,“述”的同时一定要有“评”,指出现有研究状态,仍存在哪些尚待解决的问题,讲出自己的研究有哪些探索性内容。
%   \item 研究方法:讲清论文所使用的学术研究方法。
%   \item 论文结构安排:介绍本论文的写作结构安排。
% \end{itemize}
