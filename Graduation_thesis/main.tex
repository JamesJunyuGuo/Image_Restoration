% !TeX encoding = UTF-8
% !TeX program = xelatex
% !TeX spellcheck = en_US

\documentclass[degree=master]{thuthesis}
  % 学位 degree:
  %   doctor | master | bachelor | postdoc
  % 学位类型 degree-type:
  %   academic(默认)| professional
  % 语言 language
  %   chinese(默认)| english
  % 字体库 fontset
  %   windows | mac | fandol | ubuntu
  % 建议终版使用 Windows 平台的字体编译


% 论文基本配置,加载宏包等全局配置
\input{thusetup}

\usepackage{listings}
\usepackage{xcolor}  % Required for adding colors

% Define custom colors for highlighting
\definecolor{codegreen}{rgb}{0,0.6,0}
\definecolor{codegray}{rgb}{0.5,0.5,0.5}
\definecolor{codepurple}{rgb}{0.58,0,0.82}
\definecolor{backcolour}{rgb}{0.95,0.95,0.92}

% Setup listings configuration
\lstset{
    backgroundcolor=\color{backcolour},   % choose the background color
    basicstyle=\footnotesize\ttfamily,    % size of fonts used for the code
    breaklines=true,                      % automatic line breaking only at whitespace
    captionpos=b,                         % sets the caption-position to bottom
    commentstyle=\color{codegreen},       % comment style
    keywordstyle=\color{magenta},         % keyword style
    numberstyle=\tiny\color{codegray},    % the style that is used for the line-numbers
    stringstyle=\color{codepurple},       % string literal style
    frame=single,                         % adds a frame around the code
    rulecolor=\color{black},              % if not set, the frame-color may be changed on line-breaks within not-black text (e.g. comments (green here))
    numbers=left,                         % where to put the line-numbers; possible values are (none, left, right)
    numbersep=10pt,                       % how far the line-numbers are from the code
    tabsize=4,                            % sets default tabsize to 4 spaces
    language=bash                         % the language of the code
}



\usepackage{algorithmic}
\usepackage{algorithm}
\makeatletter
\newenvironment{breakablealgorithm}
  {% \begin{breakablealgorithm}
   \begin{center}
     \refstepcounter{algorithm}% New algorithm
     \hrule height.8pt depth0pt \kern2pt% \@fs@pre for \@fs@ruled
     \renewcommand{\caption}[2][\relax]{% Make a new \caption
       {\raggedright\textbf{\ALG@name~\thealgorithm} ##2\par}%
       \ifx\relax##1\relax % #1 is \relax
         \addcontentsline{loa}{algorithm}{\protect\numberline{\thealgorithm}##2}%
       \else % #1 is not \relax
         \addcontentsline{loa}{algorithm}{\protect\numberline{\thealgorithm}##1}%
       \fi
       \kern2pt\hrule\kern2pt
     }
  }{% \end{breakablealgorithm}
     \kern2pt\hrule\relax% \@fs@post for \@fs@ruled
   \end{center}
  }
\makeatother


\begin{document}

% 封面
\maketitle



% 使用授权的说明
\copyrightpage
% 将签字扫描后授权文件 scan-copyright.pdf 替换原始页面
% \copyrightpage[file=scan-copyright.pdf]

\frontmatter
% !TeX root = ../thuthesis-example.tex

% 中英文摘要和关键字

\begin{abstract}
在当今数字图像处理与计算机视觉领域,图像去噪及图像生成是核心且至关重要的问题。近年来,机器学习方法的崛起使得高质量图像的生成成为研究焦点,其中扩散模型因其出色的生成效果而受到广泛关注。深度扩散模型,由正向扩散与逆向采样两部分构成,通过训练及采样过程,逐渐逼近目标图像的分布。

然而,在图像条件生成领域,深度学习驱动的扩散模型面临着训练难度高、表达能力有限及泛化能力不足的挑战。首先,现有模型对输入图像的噪声较为敏感,缺乏足够的鲁棒性。其次,针对特定数据集训练的扩散模型成本高昂,如何有效利用预训练集辅助条件生成成为一大难题。最后,在模型泛化方面,如何设计适用于不同图像条件生成任务(如图像修复、去模糊、超分辨率等)的高效算法,成为当前研究的热点。

针对上述问题,本文提出了一种高效的图像条件生成算法,其创新点主要体现在以下三个方面:

首先,针对深度学习模型对噪声输入不鲁棒的问题,本文提出了一种基于扩散模型的图像去噪算法,该算法通过对后验分布似然函数的逼近,实现了对输入噪声的稳定去噪。相较于传统模型,该算法不仅具有更好的表达能力,而且函数族严格包含已有模型的函数族。

其次,针对DDPM采样方法训练开销大的问题,本文利用DDIM模型中的采样方法进行分段采样,有效减少了迭代次数,从而降低了生成图片数量较大时的训练成本,使得图像修复算法得以高效实现。

最后,针对条件生成算法中采样速度慢及学习率调整复杂的问题,本文提出了一种无模型依赖的学习率设置算法。该算法经过实验验证,适用于LSUN、Imagenet及FFHQ等数据集,显著提高了条件生成算法的稳定性。

综上所述,本文提出的算法在图像条件生成领域具有显著的创新性和实用性,为解决当前存在的问题提供了新的思路和方法。
  \thusetup{
    keywords = { 图像修复 ,扩散模型, 神经网络, 后验估计, 分层模型},
  }
\end{abstract}

\begin{abstract*}
In the current field of digital image processing and computer vision, image denoising and generation have always been crucial issues. In recent years, the rise of machine learning methods has made the generation of high-quality images a research focus, with diffusion models receiving widespread attention due to their excellent generation effects. Deep diffusion models, consisting of forward diffusion and reverse sampling processes, gradually approximate the distribution of target images through training and sampling.

However, in the realm of conditional image generation, deep learning-driven diffusion models face challenges in terms of training difficulty, limited expressive power, and insufficient generalization ability. Firstly, existing models are sensitive to noise in input images and lack sufficient robustness. Secondly, the cost of training diffusion models for specific datasets is high, and how to effectively utilize pre-trained sets to assist in conditional generation has become a major challenge. Finally, in terms of model generalization, designing efficient algorithms suitable for different conditional image generation tasks such as image restoration, deblurring, and super-resolution has become a hotspot in current research.

In response to the above issues, this paper proposes an efficient algorithm for conditional image generation. The main innovations are reflected in the following three aspects:

Firstly, addressing the issue of deep learning models' lack of robustness to noisy inputs, this paper proposes an image denoising algorithm based on diffusion models. By approximating the likelihood function of the posterior distribution, the algorithm achieves stable denoising of input noise. Compared to traditional models, it not only exhibits better expressive power but also strictly contains the function families of existing models.

Secondly, to address the high training cost associated with the DDPM sampling method, this paper utilizes the sampling method from the DDIM model to perform segmented sampling, effectively reducing the number of iterations. This approach lowers the training cost when generating a large number of images, enabling efficient implementation of image restoration algorithms.

Finally, addressing the issues of slow sampling speed and the complexity of manually adjusting learning rates in conditional generation algorithms, this paper proposes a model-free learning rate setting algorithm. Experimental verification has shown that this algorithm is suitable for datasets such as LSUN, Imagenet, and FFHQ, significantly improving the stability of conditional generation algorithms.

In summary, the algorithm proposed in this paper exhibits significant innovation and practicality in the field of conditional image generation, providing new ideas and methods for addressing existing problems.

  % Use comma as separator when inputting
  \thusetup{
    keywords* = {Image Restoration, Diffusion Model, Neural Network, Posterior Estimation, Stratified Model},
  }
\end{abstract*}


% 目录
\tableofcontents


\listoffigures           % 插图清单
% \listoftables            % 附表清单

% 符号对照表
\input{data/denotation}


% 正文部分
\mainmatter

\chapter{引言}
\section{问题介绍}

在当今数字图像处理和计算机视觉领域,图像去噪一直是一个至关重要的问题。在实际应用中,由于各种因素,例如传感器噪声、压缩、低光条件等,图像中常常存在噪声,这会影响到图像的质量和后续处理任务的准确性。如何从现实中实际获取带噪图像中还原得到原先清晰度更高的原始图像具有重大的应用价值和前景。
因此,开发高效准确的图像去噪算法对于提升图像质量以及下游任务的性能具有重要意义。
\subsection*{图像去噪的挑战}
传统的图像去噪方法通常基于滤波器和统计学方法,它们可以在某些场景下取得不错的效果。然而,在面对复杂的噪声分布或者噪声与信号之间相互重叠的情况下,传统方法的性能往往受到限制。特别是对于低信噪比(SNR)的图像,传统方法可能会产生过度平滑或者细节损失的问题。      

若利用传统统计学习方法来进行图像处理,则通常将二维图像当作一维向量进行处理。图像作为一类特殊的数据类型其分布通常可以被看作在高维向量空间中的一个低维流形的嵌入,即在图像中通常有较多常见特征反复出现,使得其有相比于一般向量空间中数据分布的特殊性质,若对其向量化则会失去一部分图像数据分布独有的特征。
\subsection*{引入深度学习}
近年来,随着深度学习技术的发展,基于深度学习的图像去噪方法逐渐受到关注。其中,Variational Autoencoder(VAE)和Diffusion Model (DM)等深度生成模型因其能够学习到图像的复杂概率分布而备受瞩目。这些模型不仅能够准确地去除噪声,还可以保留图像的细节和结构,从而在实际应用中取得更好的效果。
\section{选题背景及意义}

\subsection*{选题背景}
随着数字图像在日常生活和工业应用中的普及,图像质量的要求越来越高。然而,在现实世界中,图像往往会受到各种形式的干扰和噪声,例如传感器噪声、压缩引起的伪像、低光条件下的图像模糊等。这些噪声不仅降低了图像的质量,还可能影响后续的图像分析和处理任务的准确性和可靠性。\par 
传统的图像去噪方法通常基于数学模型和信号处理技术,例如中值滤波、高斯滤波等。然而,这些方法往往只能处理特定类型的噪声,对于复杂的噪声分布或者噪声与信号之间的相互干扰,传统方法的效果可能不尽如人意,具有较大的局限性。
\subsection*{选题意义}
提升图像质量: 有效的图像去噪算法可以帮助提升图像的质量,使其更加清晰和真实,提高观感体验。\par 
改善后续处理任务的效果: 在许多图像处理和分析任务中,图像质量对结果的影响至关重要。去噪后的图像更容易进行特征提取、对象检测、图像识别等后续处理任务,从而提高了这些任务的准确性和鲁棒性。\par 
推动深度学习在图像处理领域的应用: 结合VAE和Diffusion Model等深度生成模型进行图像去噪不仅可以提高去噪效果,还可以推动深度学习技术在图像处理领域的应用。通过将深度学习引入图像去噪领域,可以充分挖掘图像数据的潜在信息,实现更加智能和高效的图像处理算法。\par 
实践应用的需求: 在实际应用中,如医学影像、卫星图像、安防监控等领域,图像质量对于决策和诊断的重要性不言而喻。因此,开发高效准确的图像去噪算法对于满足实践应用的需求具有重要意义。\par 
综上所述,研究图像去噪算法不仅有助于提升图像质量和后续处理任务的效果,还有助于推动深度学习技术在图像处理领域的应用,满足实际应用的需求,具有重要的理论和实践意义。
\subsection*{本研究的目的}
本研究旨在探索如何结合VAE和Diffusion Model这两种先进的深度生成模型来解决图像去噪问题。通过结合VAE的自编码器结构和Diffusion Model的概率建模能力,我们希望实现对于不同类型噪声的鲁棒去除,并保持图像的真实性和细节。本研究的结果不仅可以提升图像去噪的效果,还可能对于其他相关领域,如图像增强和图像重建等,产生积极的影响。通过对图像去噪问题的深入研究,我们有望为实际应用提供更加鲁棒和可靠的解决方案,推动计算机视觉技术在实践中的进一步发展。
\section{论文结构安排}
首先,在第二章进行图像修复相关文献回顾,包括传统非条件生成下的VAE模型到扩散模型,再到当今最高效的stable diffusion的原理,回顾了如何用如上无监督学习下的深度学习算法来高效学习到原数据集的总体分布。     

在第三章引入本文主要讨论主题:即图像修复的下游任务,阐述本文主要的研究目的与具体研究问题阐释。      

在第四章中,主要对实验细节和参数设定进行讨论,展示了本文在工程实现图像去噪任务中的贡献。 并且对条件生成下在不同算法下的生成图像质量进行分析比较,从而论证了本文所提出算法的优越性。     

最后,在第五章中,对图像修复任务的未来进行展望,总结了在未来可以有的提升方向。 

% \section{引言的写法}

% 一篇学位论文的引言大致包含如下几个部分:
% 1、问题的提出;
% 2、选题背 景及意义;
% 3、文献综述;
% 4、研究方法;
% 5、论文结构安排。
% \begin{itemize}
%   \item 问题的提出:要清晰地阐述所要研究的问题“是什么”。
%     \footnote{选题时切记要有“问题意识”,不要选不是问题的问题来研究。}
%   \item 选题背景及意义:论述清楚为什么选择这个题目来研究,即阐述该研究对学科发展的贡献、对国计民生的理论与现实意义等。
%   \item 文献综述:对本研究主题范围内的文献进行详尽的综合述评,“述”的同时一定要有“评”,指出现有研究状态,仍存在哪些尚待解决的问题,讲出自己的研究有哪些探索性内容。
%   \item 研究方法:讲清论文所使用的学术研究方法。
%   \item 论文结构安排:介绍本论文的写作结构安排。
% \end{itemize}

\chapter{文献回顾}
本文的最终目的是处理条件生成下的图像修复问题,即在给定测量下的加工后图像,力求还原在原有数据集中极大似然意义下的原像。例如在图像\ref{original image }作为原始图像,如果去除图像中的部分像素点得到损坏图像\ref{inpainted image}。则我们的任务是已知原图像是属于FFHQ数据集,我们想要通过损坏图像\ref{inpainted image}来还愿得到最有可能属于原数据集的原像图像。    

在该图像修复问题中,涉及一下两个难点:     

1. 如何得到关于原数据集的信息,在如上例子中即为,如何获取FFHQ数据集的特征并确保最终生成图像是属于该数据集的。   

2. 如何在给定损坏图像的情形下,还原得到在极大似然估计意义下该损坏图像的原像,即需要定义何为最佳原像。    

为了应对上述两个核心问题,本章首先综述了无条件生成模型的相关文献,这一过程旨在掌握原始数据集分布的关键信息。随后,本文还总结了当前条件生成模型在图像修复领域的研究成果和贡献。
\begin{figure}[H]
  \centering
  \begin{minipage}[b]{0.45\linewidth}
    \includegraphics[width=\linewidth]{ThuThesis_ Tsinghua University Thesis LaTeX Template/figures/intro/input.png}
    \caption{原始图像}
    \label{original image }
  \end{minipage}
  \hspace{0.5cm} % Space between images
  \begin{minipage}[b]{0.45\linewidth}
    \includegraphics[width=\linewidth]{ThuThesis_ Tsinghua University Thesis LaTeX Template/figures/intro/label.png}
    \caption{损坏图像}
    \label{inpainted image}
  \end{minipage}
\end{figure}
\section{无条件生成模型}
\subsection{变分编码器(VAE Model)}
变分编码器(Variational Auto Encoder Model)是一个有效学习未知高维数据分布的生成式模型,广泛应用于图像,音频,视频等数据用用于生成相似分布的数据。在诸多Bayes模型中,通常需要构造隐式变量(Latent Variable)去用来刻画目标分布,然而通常对于隐式变量的后验分布是没有任何额外信息的,同时在许多模型下也不存在解析表达式。\par 
首先明确VAE模型的适用问题范围,VAE模型主要用来处理i.i.d型数据点样本,利用变分推断等方法获得对该连续分布数据的逼近。
在VAE模型中,引入了编码器(Encoder)用来逼近隐式变量的后验分布(基于现有样本),利用变分推断、极大似然估计等技术来训练解码器(Decoder),从而利用隐式变量生成对目标样本分布的逼近。以下为变分编码器构成的图例,在图\ref{VAE model fig}中,首先中,首先将原始数据点作为输入,利用编码器可生成隐式变量。通过在隐式变量空间中重新采样,可以重新通过解码器映射到样本空间中,生成对原样本分布的逼近。
\begin{figure}[H]
    \centering
    \includegraphics[scale = 0.7]{ThuThesis_ Tsinghua University Thesis LaTeX Template/Picture/VAE.png}
    \caption{VAE 模型构成}
    \label{VAE model fig}
\end{figure}
以上为VAE模型的总体架构说明,接下来的部分将对VAE模型的数学原理进行说明。VAE模型同传统贝叶斯模型一样,同样采用极大似然估计的方法(MLE)对参数进行估计。不同点在于不直接对似然函数进行优化,采用变分推断的方法对对数似然函数的相反数下界进行极小化。该方法避免了不存在解析解的困境,易于数值计算提升优化效率。接下来定义相关符号,在VAE模型中,解码器为$p_{\theta}(x|z)$,在给定了隐式变量$z$的情况下通过以$\theta$为参数化的模型定义了一个条件概率分布。编码器$q_{\phi}(z|x)$,用来逼近真实的后验分布$p_{\theta}(z|x)$。与传统的Bayes模型不同的地方在于,编码器和解码器不直接定义从样本空间和隐式空间之间的具体映射,而是从一个空间的样本定义在另一个空间内样本的分布,再通过采样的方式获得。假设给定样本点为$\{x^1,\cdots,x^{N}\}$,在MLE的原则下,我们的目标是极大化$\mathbb{E}[\log(p_{\theta}(x))]$,
\begin{equation}
    \mathbb{E}[\log(p_{\theta}(x))] = \operatorname{KL}(q_{\phi}(z|x)||p_{\theta}(z|x)) + \mathcal{L}(x,\theta,\phi)
\end{equation}
其中$\mathcal{L}$可以写为
\begin{align}
    \mathcal{L}(x,\theta,\phi) &= \mathbb{E}_{q_{\phi}(z|x)}\left[-\log(q_{\phi}(z|x))+\log(p_{\theta}(z,x))\right]\\
    &=-\operatorname{KL}(q_{\phi}(z|x)||p_{\theta}(z))+\mathbb{E}_{q_{\phi}(z|x)}\left[p_{\theta}(x|z))\right]
\end{align}
VAE模型中一个重要假设就是需要隐式变量$z$所服从的先验分布$p_{\theta}(z)$为正态分布,则KL-散度在两个分布均为正态分布的条件下可以有解析表达式,方便在后续过程中直接得到梯度。在此处$\mathcal{L}$即为对数似然函数的一个下界,通过最大化$\mathcal{L}$我们相当于可以不断最大化似然函数(可以不断调整$\phi$使得$q_{\phi}(z|x)$和真正的后验分布$p_{\theta}(z|x)$越来越接近)。此处我们同样可以称$\mathcal{L}$为ELBO(即Evidence Lower Bound),整个VAE模型通过最大化ELBO来进行参数优化,即
\begin{align}
\operatorname{ELBO}&=\mathbb{E}_{q_{\phi}(z|x)}\big[\log(\frac{p_{\theta}(z,x)}{q_{\phi}(z|x)})\big]\\
    \mathcal{L}(x,\theta, \phi)&=\mathbb{E}_{q_\phi(z \mid x)}\left[\log p_\theta(x \mid z)\right]-\mathcal{D}_{K L}\left[q_\phi(z \mid x) \| p(z)\right]
    \end{align}
在进行具体优化的时候,还需要用到重参数化技术(Reparameterization Trick),优化ELBO的时候需要分别得到其关于$\phi$和$\theta$的梯度。由于在VAE的假设中隐含了$p(z)\sim \mathcal{N}(z;0,I)$,因此$KL(q_{\phi}(z|x)||p(z))$为关于$\phi$的多项式函数,可以直接求导得到梯度,主要问题集中于得到$\mathbb{E}_{q_{\phi}(z|x)}\left[p_{\theta}(x|z))\right]$分别关于$\theta$,$\phi$的梯度。我们有
\begin{equation}
    \nabla_{\theta}\mathbb{E}_{q_{\phi}(z|x)}\left[p_{\theta}(x|z))\right] = \mathbb{E}_{q_{\phi}(z|x)}\left[\nabla_{\theta}p_{\theta}(x|z))\right]
\end{equation}
但是一般而言,
\begin{equation}
    \nabla_{\phi}\mathbb{E}_{q_{\phi}(z|x)}\left[p_{\theta}(x|z))\right]\neq \mathbb{E}_{q_{\phi}(z|x)}\left[\nabla_{\phi}p_{\theta}(x|z))\right]
\end{equation}
此处可利用重参数化方法,假设$z\sim q_{\phi}(z|x)=\mathcal{N}(z;\mu_{\phi}(x),\Sigma_{\phi}(x))$,则可以将$z$写为$z = \mu_{\phi}(x) +\Sigma_{\phi}(x)^{1/2} \epsilon $,其中$\epsilon \sim \mathcal{N}(\epsilon,0,I)$, 为标准正态分布。则可以将原式写为
\begin{align}
    \nabla_{\phi}\mathbb{E}_{q_{\phi}(z|x)}\left[p_{\theta}(x|z))\right]&=\nabla_{\phi}\mathbb{E}_{z\sim q(x,\phi,\epsilon)}\left[p_{\theta}(x|z))\right]\\
    &= \nabla_{\phi}\mathbb{E}_{p(\epsilon)}\left[p_{\theta}(x|q(x,\phi,\epsilon)))\right]= \mathbb{E}_{p(\epsilon)}\left[\nabla_{\phi} p_{\theta}(x|q(x,\phi,\epsilon)))\right]
\end{align}
由此可以得到关于$\phi$的梯度,以上便为VAE模型的优化流程。
\subsection{正则流}
正则流(Normalization Flow)是基于变分编码器的结构基础上提出的(见Improving Variational Auto-Encoders
using Householder Flow)用来逼近真实的隐藏变量的先验分布$p_{\theta}(z)$。首先先由样本$x$生成$z_0$的简单分布
(一般设为正态分布,由$x$生成$z_0$分布的均值和方差)。然后,对$z_0$进行一系列可逆变换$f^{(t)},t=1,2,\dots,T$。正则流可以将初始的密度函数通过一系列可逆的连续函数变成更加复杂的密度函数(由于实际中的隐式变量的先验分布不一定服从标准正态分布,因此通过正则流转换为实际的复杂分布)。 一旦我们选定了转换函数
$f^{(t)}$, 我们可以计算出其雅可比矩阵行列式。该方法用到的重要假设为,任何两个连续分布$X_1\sim \mathcal{D}_1,X_2\sim \mathcal{D}_2$,均存在一个连续函数$f$使得$f(X_1)\sim \mathcal{D_2}$。正则流利用函数的复合去逼近得到该函数$f$从而使得映射得到的隐式变量的先验分布服从标准正态分布。其局限性在于仅仅能够对服从连续分布的随机变量成立,无法适用于离散型数据。以下为正则流的图例。
\begin{figure}[H]
    \centering
    \includegraphics[scale=0.15]{ThuThesis_ Tsinghua University Thesis LaTeX Template/Picture/norm_flow.png}
    \caption{正则流图例}
    \label{Norm_flow}
\end{figure}
对于连续函数$f:\mathbb{R}^{d}\rightarrow \mathbb{R}^d$,其中$f$可逆,定义$f$逆映射$f^{-1}=g$,即$g\circ f(z)=z$。如果我们用该映射将具有密度函数$q(z)$的隐式随机变量$z$映射到$z^{\prime}=f(z)$,则$z'$有如下密度函数
\begin{equation}
    q(z^{\prime}) = q(z)|\operatorname{det}\frac{\partial f^{-1}}{\partial z^{\prime}}|= q(z)|\operatorname{det}\frac{\partial f}{\partial z}|^{-1},
    \label{chain rule}
\end{equation}
其中最后一个等式可以利用链式法则来进行推广(如果有多个函数复合的情况下),我们可以复合多个函数并且循环利用式(\ref{chain rule})。对于有$K$个映射作用下得到的随机变量$z_{K}$(如图\ref{Norm_flow}中所示),由随机变量$z_0$连续变换$K$次得到的随机变量的密度函数$q_{K}(z)$具有性质
\begin{equation}
    \ln q_{K}(z_{K}) = \ln q_{0}(z_0)-\sum_{k=1}^{K}\ln |\operatorname{det}\frac{\partial f_k}{\partial z_{k-1}}|
    \label{chain log likelihood}
\end{equation}
其中我们有
\begin{equation}
    z_{K}=f_{K}\circ \cdots f_1(z_0)
    \label{z_K definition}
\end{equation}
根据VAE模型的定义,我们的优化目标转换为如下目标:
\begin{equation}
    \ln p(\mathbf{x}) \geq \mathbb{E}_{q\left(\mathbf{z}^{(0)} \mid \mathbf{x}\right)}\left[\ln p\left(\mathbf{x} \mid \mathbf{z}^{(T)}\right)+\sum_{t=1}^T \ln \left|\operatorname{det} \frac{\partial \mathbf{f}^{(t)}}{\partial \mathbf{z}^{(t-1)}}\right|\right]-\mathrm{KL}\left(q\left(\mathbf{z}^{(0)} \mid \mathbf{x}\right) \| p\left(\mathbf{z}^{(T)}\right)\right)
    \label{NF objective}
    \end{equation}
正则流可以用来逼近不同类型的VAE模型中的后验分布,只需要选取合适的转换函数$f^{(t)}$,不需要对编码器和解码器的结构进行修改。在实际应用中,一般采取形式较为简单的可逆正则流(对每个元素进行运算而减少矩阵乘法的操作,从而不需要储存矩阵的逆提升计算速率)。以下列举代表性正则流例子。
\subsection{线性时间转换正则流模型(Invertible Linear-time Transformation)}
我们考虑如下形式的可逆变换
\begin{equation}
    f(z) = z+ u\cdot h(w^{\top } w + b)
    \label{family form}
\end{equation}
其中我们定义参数集$\lambda = \{w\in \mathbb{R}^{D}, u\in \mathbb{R}^D,b\in \mathbb{R}\}$ 为可以调节的参数,$h(\cdot)$为一个光滑非线性函数有导数$h'(\cdot)$,该映射对矩阵每一个元素进行相同的映射,为一个elementwise function。对于此类映射我们可以在$O(D)$复杂度时间内计算出对数雅可比项,计算如下
\begin{align}
    \psi(z) &= h'(w^{\top}z+b)w\\
    |\operatorname{det}\frac{\partial f}{\partial z}| &= |\operatorname{det}(I+u\cdot \psi(z)^{\top})|=|1+u^{\top}\psi(z)|
\end{align}
根据式(\ref{chain log likelihood})可以得到最终随机变量的似然函数以及根据在式(\ref{family form})中定义的正则流转换函数的形式可以得到在该族转换函数下的似然函数取值形式为
\begin{align}
     z_{K} &=f_{K}\circ \cdots f_1(z_0) \\
     \ln q_{K}(z_{K})&= \ln q_0(z)-\sum_{k=1}^{K}\ln |1+u_{k}^{\top}\psi_{k}(z_{k-1})|
\end{align}
\subsection{扩散模型}
扩散概率模型在图像生成领域已经被证明是有效的生成模型,但是其缺点在于缺乏低维的有效隐藏变量$z$。然而,VAE模型通常具有低维有效的隐藏变量,可以通过结合两者来更加有效地
获得对高维数据分布的逼近。
和VAE模型一样,扩散模型的目的也是希望能够学习得到目标高维样本的分布性质。 在\cite{diffusion}中详细讨论了扩散模型的性质并且介绍了扩散模型的特殊情况,即DDPM模型用来对实际应用中的高维数据进行处理。以下部分中,我们首先将对扩散模型的数学原理进行回顾。基于随机微分方程的背景,假设有一个如下的随机微分方程
\begin{equation}
    dz = f(t)\cdot z \rm{dt}+g(t)\rm{dw}.
    \label{SDE 1}
\end{equation}
其中$\{z_t\}_{t=0}^{t=1}$为一个正向的连续变化的扩散过程,其中$z_0$为初始的随机变量,在此处我们将其作为想要研究的为止分布,而$z_t$则为在$t$时刻扩散得到的随机变量。其中, $f: \mathbb{R}\rightarrow \mathbb{R}$与
 $g: \mathbb{R}\rightarrow \mathbb{R}$为一维函数,$w$为标准布朗运动。可以通过设计$f$和$g$使得$z_1$服从标准正态分布,即$z_1\sim \mathcal{N}(z_1;0,I)$。而在实际对该连续变化过程进行采样的时候,通常会使用Euler法,即将$[0,1]$区间分成$T$个子区间$[i/T,(i+1)/T], i =0,1,\cdots, (T-1)$, 仅关注每一个子区间端点处随机变量的分布,即为$z_{i/T},i=0,1,\cdots,T$。在\cite{diffusion}中已经证明在式(\ref{SDE 1})中的随机微分方程可以通过一个生成式模型从尾端进行反转。在该文章中,在保证$z_1$服从标准正态分布的前提假设下,可以先对$z_1\sim \mathcal{N}(z_1;0,I)$进行取样,其次对逆向随机微分方程 $dz = \left[f(t)\cdot z -g(t)^2 \nabla_z \log q_t(z)\right] \rm{dt} + g(t)d \Tilde{w}$进行求解。其中$\Tilde{w}$为逆向标准布朗运动。在该逆向随机微分方程中,需要知道$\nabla_z \log q_t(z)$,即为在$t$时刻随机变量的正向传播下的边缘分布的打分函数(score function)。
 在\cite{diffusion}中训练逆向传播过程的参数只需要最小化如下量用来匹配打分函数
\begin{equation}
    \min _{\boldsymbol{\theta}} \mathbb{E}_{t \sim \mathcal{U}[0,1]}\left[\lambda(t) \mathbb{E}_{q\left(\mathbf{z}_0\right)} \mathbb{E}_{q\left(\mathbf{z}_t \mid \mathbf{z}_0\right)}\left[|| \nabla_{\mathbf{z}_t} \log q\left(\mathbf{z}_t\right)-\nabla_{\mathbf{z}_t} \log p_{\boldsymbol{\theta}}\left(\mathbf{z}_t\right) \|_2^2\right]\right]
    \label{score objective 1}
\end{equation}
在此处我们还需要定义权重函数$\lambda(t)$。其中$q(z_0)$为目标样本的分布函数以及$q(z_t|z_0)$是扩散核,通过选取适当的$f(t)$和$g(t)$可以使得$q(z_t|z_0)$有解析表达式。在一般设定下由于不能保证有解析表达式,在\cite{diffusion}中将式(\ref{score objective 1})转换为优化如下目标
\begin{equation}
    \min _{\boldsymbol{\theta}} \mathbb{E}_{t \sim \mathcal{U}[0,1]}\left[\lambda(t) \mathbb{E}_{q\left(\mathbf{z}_0\right)} \mathbb{E}_{q\left(\mathbf{z}_t \mid \mathbf{z}_0\right)}\left[|| \nabla_{\mathbf{z}_t} \log q\left(\mathbf{z}_t \mid \mathbf{z}_0\right)-\left.\nabla_{\mathbf{z}_t} \log p_{\boldsymbol{\theta}}\left(\mathbf{z}_t\right)\right|_2 ^2\right]\right]+C
    \label{score objective 2}
\end{equation}
其中$C=\mathbb{E}_{t \sim \mathcal{U}[0,1]}\left[\lambda(t) \mathbb{E}_{q\left(\mathbf{z}_0\right)} \mathbb{E}_{q\left(\mathbf{z}_t \mid \mathbf{z}_0\right)}\left[\left\|\nabla_{\mathbf{z}_t} \log q\left(\mathbf{z}_t\right)\right\|_2^2-\left\|\nabla_{\mathbf{z}_t} \log q\left(\mathbf{z}_t \mid \mathbf{z}_0\right)\right\|_2^2\right]\right]$,在
\cite{vincent}中说明了$C$的取值不取决于$\theta$,因此由式(\ref{score objective 2})和式(\ref{score objective 1})优化得到的参数$\theta$是等价的。为了计算上的便捷,通常会取$\lambda(t) = g(t)^2/2$, 其中在\cite{song_2}中说明了如果以如上方式选取权重函数$\lambda(t)$则可以使得在KL-散度意义下的目标分布和逆向SDE生成的分布的距离可以以式(\ref{score objective 1})作为上界,即
\begin{equation}
\operatorname{KL}\left(q\left(\mathbf{z}_0\right) \| p_{\boldsymbol{\theta}}\left(\mathbf{z}_0\right)\right) \leq \mathbb{E}_{t \sim \mathcal{U}[0,1]}\left[\frac{g(t)^2}{2} \mathbb{E}_{q\left(\mathbf{z}_0\right)} \mathbb{E}_{q\left(\mathbf{z}_t \mid \mathbf{z}_0\right)}\left[\left\|\nabla_{\mathbf{z}_t} \log q\left(\mathbf{z}_t\right)-\nabla_{\mathbf{z}_t} \log p_{\boldsymbol{\theta}}\left(\mathbf{z}_t\right)\right\|_2^2\right]\right]
    \label{KL upper bound 1}
\end{equation}
在实际优化模型中,只需要对式(\ref{score objective 2})进行最小化即可,在得到相应模型参数$\theta$后对逆向随机微分方程进行求解可以得到对目标分布的逼近。由于扩散模型只需要保证末端的随机变量$z_1$服从正态分布,因此可以选取不同的$f(t)$,$g(t)$函数来生成目标分布。在接下来的部分中将着重分析一类特殊的扩散模型:DDPM模型。(Denoising Diffusion Probabilistic Models)
\subsection{降噪扩散概率模型(DDPM)}
在本部分将重新引入新的符号,这里我们定义$x_i= z_{i/T},i=0,1,\cdots,T$。
扩散模型为隐式变量模型满足如下形式$p_{\theta}(x_0) = \int p_{\theta}(x_{0:T}) dx_{1:T}$。其中$x_0=z_0\sim q(z_0)$为样本中的原始分布,$x_1,\cdots, x_{T}$ 为和$x_0$相同维数的隐式变量。 $x_T$对应一般扩散模型中的$z_1$,满足服从标准正态分布。由于$x_{0:T}$分别对应在$[0,1]$区间中的通过Euler法得到格点上的随机变量,由于此处可以假设$T$足够大,因此每一个相邻的条件概率分布$q(x_{i+1}|x_i)$都服从正态分布(利用布朗运动的性质)。\par 
首先开始定义正向过程,在正向过程中,扩散模型采用一步步向样本增加白噪声的方式,使得最终获得的随机变量$x_T$的分布趋近于正态分布。我们在此处通过选取参数$\beta_1, \cdots, \beta_{T}$ 的方式来选取每次增加噪声的大小。扩散模型的目的在于先用正向过程从$x_0$过渡到$x_T$使得其分布趋近于正态分布,再利用参数化得到的逆向过程来生成对初始样本分布的逼近。仿照逆向过程联合分布函数,正向联合分布函数的形式可以写为如下的形式:($\beta_t$用来定义相邻的条件概率分布的参数)
\begin{equation}
    q\left(x_{1: T} \mid x_0\right)=\prod_{t=1}^T q\left(x_t \mid x_{t-1}\right)
    \end{equation}
    \begin{equation}
        q\left(x_t \mid x_{t-1}\right)=\mathcal{N}\left(\sqrt{1-\beta_t} x_{t-1}, \beta_t I\right)
        \end{equation}
        \begin{equation}
            q\left(x_t \mid x_0\right)=\mathcal{N}\left(\sqrt{\bar{\alpha}_t} x_0,\left(1-\overline{\alpha_t}\right) I\right) \text { 其中 } \alpha_t=\left(1-\beta_t\right) \text { 以及 } \bar{\alpha}_t=\prod_t \alpha_t
            \label{posterior of xt}
            \end{equation}
    正向传播的后验分布同样可以被给出:
    \begin{equation}
        q\left(x_{t-1} \mid x_t, x_0\right)=\mathcal{N}\left(\tilde{\mu}_t\left(x_t, x_0\right), \tilde{\beta}_t\right)
        \label{posterior xt 2}
        \end{equation}
        \begin{equation}
            \text {     其中 } \tilde{\mu}_t\left(x_t, x_0\right)=\frac{\sqrt{\bar{\alpha}_{t-1}} \beta_t}{1-\bar{\alpha}_t} x_0+\frac{\sqrt{\alpha_t}\left(1-\bar{\alpha}_{t-1}\right)}{1-\bar{\alpha}_t} x_t
            \end{equation}
            \begin{equation}
                \text { 以及 } \quad \tilde{\beta}_t=\frac{1-\bar{\alpha}_{t-1}}{1-\bar{\alpha}_t} \beta_t
                \end{equation}
该正向过程实际上对应的随机微分方程即为
\begin{equation}
    dz = -\frac{g(t)^2}{2}\cdot  z dt + g(t)dw
\end{equation}
$x_{0:T}$的联合分布$p_{\theta}(x_{0:T})$被称为逆向过程,并且它被定义为一个Markov过程,该过程的转移分布可以从末端$p(x_T)= \mathcal{N}(x_T;0,I) $开始。由此可以写出逆向过程的联合分布函数的表达式
\begin{equation}
    p_{\theta}(x_{0:T}) := p(x_T)\prod_{t=1}^{T}p_{\theta}(x_{t-1}|x_t), p_{\theta}(x_{t-1}|x_t) :=\mathcal{N}(x_{t-1};\mu_{\theta}(x_t,t),\Sigma_{\theta}(x_t,t)).  
\end{equation}

以上为正向Markov过程所满足的条件概率分布性质,在随机优化问题中为了优化参数需要对逆向传播过程进行建模。
逆向传播过程同样可以被参数化,此处使用高斯转移分布来进行此一阶Markov逆向传播过程,满足如下表达式:
\begin{equation}
    p\left(x_{0: T}\right)=p\left(x_T\right) \prod_{t=1}^T p_\theta\left(x_{t-1} \mid x_t\right)
    \end{equation}
    \begin{equation}
        p_\theta\left(x_{t-1} \mid x_t\right)=\mathcal{N}\left(\mu_\theta\left(x_t, t\right), \Sigma_\theta\left(x_t, t\right)\right)
        \end{equation}
        \begin{figure}[H]
            \centering
            \includegraphics[scale = 0.2]{ThuThesis_ Tsinghua University Thesis LaTeX Template/Picture/Diffusion.png}
            \caption{扩散模型图例}
            \label{Diffusion}
        \end{figure}
选取足够大的$T$值与合适的递减序列$\{\beta_t\}_{t\geq 1}$, 条件概率分布$q(x_{T}|x_0)$可以足够接近标准高斯分布。
整个概率系统可以通过变分推断来进行端到端的训练,从而优化参数。
在该模型下,同样采用极大似然估计的方式来进行参数优化,此处采用最大化对数似然函数的相反数的上界来进行优化,即
\begin{equation}
    \mathbb{E}\left[-\log p_\theta\left(\mathbf{x}_0\right)\right] \leq \mathbb{E}_q\left[-\log \frac{p_\theta\left(\mathbf{x}_{0: T}\right)}{q\left(\mathbf{x}_{1: T} \mid \mathbf{x}_0\right)}\right]=\mathbb{E}_q\left[-\log p\left(\mathbf{x}_T\right)-\sum_{t \geq 1} \log \frac{p_\theta\left(\mathbf{x}_{t-1} \mid \mathbf{x}_t\right)}{q\left(\mathbf{x}_t \mid \mathbf{x}_{t-1}\right)}\right]=: L
    \label{DDPM objective 1}
\end{equation}
从随机优化方法的角度去考虑,可以将式(\ref{DDPM objective 1})重新写为
\begin{equation}
\mathbb{E}_q[\underbrace{\mathcal{D}_{K L}\left(q\left(x_T \mid x_0\right) \| p\left(x_T\right)\right)}_{L_T}+\sum_{t>1} \underbrace{\mathcal{D}_{K L}\left(q\left(x_{t-1} \mid x_t, x_0\right) \| p_\theta\left(x_{t-1} \mid x_t\right)\right)}_{L_{t-1}}-\underbrace{\log p_\theta\left(x_0 \mid x_1\right)}_{L_0}]
    \end{equation}
以上为DDPM的数学原理,以下将具体讨论DDPM在实际优化过程中的参数设计与模型细节。以上已经将$L$分解成$L_T,L_{t-1},L_0$三大部分,由于后验分布$q$没有任何参数,因此$L_T$可以直接忽略(和$\theta$取值近似无关)。
在\cite{DDPM}中直接假设$\Sigma_{\theta}(x_t,t)=\sigma_t^2 I$,方便简化计算过程(避免矩阵求逆运算操作)。其次,为了表示$\mu_{\theta}(x_t,t)$,由于$p_{\theta}(x_{t-1}|x_t):= \mathcal{N}(x_{t-1};\mu_{\theta}(x_t,t),\sigma_t^2I)$,我们可以将$L_{t-1}$写为
\begin{equation}
    L_{t-1}=\mathbb{E}_q\left[\frac{1}{2 \sigma_t^2}\left\|\tilde{\boldsymbol{\mu}}_t\left(\mathbf{x}_t, \mathbf{x}_0\right)-\boldsymbol{\mu}_\theta\left(\mathbf{x}_t, t\right)\right\|^2\right]+C
    \label{Lt representation}
\end{equation}
其中$C$为一个和$\theta$无关的常数,在优化过程中可以被忽略。利用式(\ref{posterior of xt})可以进一步简化式(\ref{Lt representation}) 将$x_t$表示为$\mathbf{x}_t\left(\mathbf{x}_0, \boldsymbol{\epsilon}\right)=\sqrt{\bar{\alpha}_t} \mathbf{x}_0+\sqrt{1-\bar{\alpha}_t} \boldsymbol{\epsilon} \text { for } \boldsymbol{\epsilon} \sim \mathcal{N}(\mathbf{0}, \mathbf{I}) $, 以及利用式(\ref{posterior xt 2})可以得到
\begin{align}
    L_{t-1}-C&=\mathbb{E}_{\mathbf{x}_0, \boldsymbol{\epsilon}}\left[\frac{1}{2 \sigma_t^2}\left\|\tilde{\boldsymbol{\mu}}_t\left(\mathbf{x}_t\left(\mathbf{x}_0, \boldsymbol{\epsilon}\right), \frac{1}{\sqrt{\bar{\alpha}_t}}\left(\mathbf{x}_t\left(\mathbf{x}_0, \boldsymbol{\epsilon}\right)-\sqrt{1-\bar{\alpha}_t} \boldsymbol{\epsilon}\right)\right)-\boldsymbol{\mu}_\theta\left(\mathbf{x}_t\left(\mathbf{x}_0, \boldsymbol{\epsilon}\right), t\right)\right\|^2\right]\\
&=\mathbb{E}_{\mathbf{x}_0, \boldsymbol{\epsilon}}\left[\frac{1}{2 \sigma_t^2}\left\|\frac{1}{\sqrt{\alpha_t}}\left(\mathbf{x}_t\left(\mathbf{x}_0, \boldsymbol{\epsilon}\right)-\frac{\beta_t}{\sqrt{1-\bar{\alpha}_t}} \boldsymbol{\epsilon}\right)-\boldsymbol{\mu}_\theta\left(\mathbf{x}_t\left(\mathbf{x}_0, \boldsymbol{\epsilon}\right), t\right)\right\|^2\right]
\end{align}
\subsection{去噪扩散隐式模型(DDIM)}
在DDPM的基础上,针对原模型所存在的缺陷,提出了DDIM,即去噪扩散隐式模型。在\cite{DDIM}中,针对DDPM所存在的缺陷,即每次进行优化需要生成整条Markov链需要较大计算复杂度的缺陷,提出了DDIM,即通过非Markov过程生成整条链。通过对比DDIM和DDPM的训练效果,发现DDIM可以生成更高质量的图像。接下来详细阐述DDIM模型正向生成样本的过程。
在正向传播过程中,Markov链的似然函数可以被写成如下形式:
\begin{equation}
    q_\sigma\left(\boldsymbol{x}_{1: T} \mid \boldsymbol{x}_0\right):=q_\sigma\left(\boldsymbol{x}_T \mid \boldsymbol{x}_0\right) \prod_{t=2}^T q_\sigma\left(\boldsymbol{x}_{t-1} \mid \boldsymbol{x}_t, \boldsymbol{x}_0\right)
\end{equation}
其中我们还有 $q_\sigma\left(\boldsymbol{x}_T \mid \boldsymbol{x}_0\right)=\mathcal{N}\left(\sqrt{\alpha_T} \boldsymbol{x}_0,\left(1-\alpha_T\right) \boldsymbol{I}\right)$ 以及对与所有的$t>1$,我们还有
$$
q_\sigma\left(\boldsymbol{x}_{t-1} \mid \boldsymbol{x}_t, \boldsymbol{x}_0\right)=\mathcal{N}\left(\sqrt{\alpha_{t-1}} \boldsymbol{x}_0+\sqrt{1-\alpha_{t-1}-\sigma_t^2} \cdot \frac{\boldsymbol{x}_t-\sqrt{\alpha_t} \boldsymbol{x}_0}{\sqrt{1-\alpha_t}}, \sigma_t^2 \boldsymbol{I}\right)
$$
以及对于每个$t$,可以选择合适的均值函数,使得$q_\sigma\left(\boldsymbol{x}_t \mid \boldsymbol{x}_0\right)=\mathcal{N}\left(\sqrt{\alpha_t} \boldsymbol{x}_0,\left(1-\alpha_t\right) \boldsymbol{I}\right)$。
根据贝叶斯公式,我们可以得到实际的条件分布密度函数为
\begin{equation}
   q_\sigma\left(\boldsymbol{x}_t \mid \boldsymbol{x}_{t-1}, \boldsymbol{x}_0\right)=\frac{q_\sigma\left(\boldsymbol{x}_{t-1} \mid \boldsymbol{x}_t, \boldsymbol{x}_0\right) q_\sigma\left(\boldsymbol{x}_t \mid \boldsymbol{x}_0\right)}{q_\sigma\left(\boldsymbol{x}_{t-1} \mid \boldsymbol{x}_0\right)}
   \label{Bayes Formula 1}
\end{equation}
根据式(\ref{Bayes Formula 1})可以得到如上分布依然为高斯分布。在该正向生成过程中,该链已经不满足Markov性质了。每次通过$\boldsymbol{x}_t,\boldsymbol{x}_0$生成$\boldsymbol{x}_{t+1}$。接下来阐述反向传播链条的具体表达形式。首先,基于$\boldsymbol{x}_t$可以得到对于$\boldsymbol{x}_0$的一个估计,其次再利用两者的结合去预测$\boldsymbol{x}_{t-1}$。对于某个$\boldsymbol{x}_0\sim q(\boldsymbol{x}_0)$以及$\epsilon_t \sim \mathcal{N}(0,I)$,函数$\epsilon_{\theta}^{(t)}(\boldsymbol{x}_t)$旨在通过$\boldsymbol{x}_t$去预测$\epsilon_t$。通过函数$f^{(t)}_{\theta}$来进行对$\boldsymbol{x}_0$的预测。
\begin{equation}
f_\theta^{(t)}\left(\boldsymbol{x}_t\right):=\left(\boldsymbol{x}_t-\sqrt{1-\alpha_t} \cdot \epsilon_\theta^{(t)}\left(\boldsymbol{x}_t\right)\right) / \sqrt{\alpha_t}.
\end{equation}
如下可以定义反向传播过程,
\begin{equation}
 p_\theta^{(t)}\left(\boldsymbol{x}_{t-1} \mid \boldsymbol{x}_t\right)= \begin{cases}\mathcal{N}\left(f_\theta^{(1)}\left(\boldsymbol{x}_1\right), \sigma_1^2 \boldsymbol{I}\right) & \text { if } t=1 \\ q_\sigma\left(\boldsymbol{x}_{t-1} \mid \boldsymbol{x}_t, f_\theta^{(t)}\left(\boldsymbol{x}_t\right)\right) & \text { 其余情形 }\end{cases}   
\end{equation}
通过变分推断的技术,可以计算得到ELBO的值,最后可以通过优化如下目标来得到参数$\theta$
\begin{align} & J_\sigma\left(\epsilon_\theta\right):=\mathbb{E}_{\boldsymbol{x}_{0: T} \sim q_\sigma\left(\boldsymbol{x}_{0: T}\right)}\left[\log q_\sigma\left(\boldsymbol{x}_{1: T} \mid \boldsymbol{x}_0\right)-\log p_\theta\left(\boldsymbol{x}_{0: T}\right)\right] \\ = & \mathbb{E}_{\boldsymbol{x}_{0: T} \sim q_\sigma\left(\boldsymbol{x}_{0: T}\right)}\left[\log q_\sigma\left(\boldsymbol{x}_T \mid \boldsymbol{x}_0\right)+\sum_{t=2}^T \log q_\sigma\left(\boldsymbol{x}_{t-1} \mid \boldsymbol{x}_t, \boldsymbol{x}_0\right)-\sum_{t=1}^T \log p_\theta^{(t)}\left(\boldsymbol{x}_{t-1} \mid \boldsymbol{x}_t\right)-\log p_\theta\left(\boldsymbol{x}_T\right)\right]\end{align}
在实际优化中,不需要每次生成T个样本,可以只选取部分样本根据条件概率分布进行生成再进行优化。
\subsection{VAE模型和扩散模型的结合:隐式扩散模型}
结合扩散模型的优点,可以去除在VAE模型中隐式变量$z$的先验分布为标准正态分布的假设。在得到隐式变量$z$后,先通过一个扩散模型学习到$p_{\theta}(z_0)$用来逼近$z_0$的真实分布,然后再通过解码器生成$x$的分布。接下来我们对该混合模型的原理进行具体回顾,在\cite{VAE_diffusion}中对该方法进行了详细的阐述。对于传统VAE模型,隐含假设为隐式变量$z$的先验分布为标准正态分布,然而该假设对于现实生活中许多具有离散分布的数据并不成立。因此需要通过扩散模型去学习到真实的隐式变量的分布,接下来详细说明该模型的执行流程。
在该模型下,我们依然只考虑如下形式的随机微分方程进行扩散
\begin{equation}
    dz = f(t)\cdot zdt + g(t)dW_t
    \label{SDE form }
\end{equation}
其中$\{W_t\}_{t\geq 0}$为标准布朗运动
\begin{figure}[H]
    \centering
    \includegraphics[scale = 0.5]{ThuThesis_ Tsinghua University Thesis LaTeX Template/Picture/VAE_diffusion.png}
    \caption{VAE模型和扩散模型的结合}
    \label{fig_VAE_Diffusion}
\end{figure}
在图\ref{fig_VAE_Diffusion}中,由三部分组成:编码器$q_{\phi}(z_0|x)$, 扩散模型$p_{\theta}(z_0)$和解码器$p_{\psi}(x|z_0)$。首先样本$x$利用编码器得到隐式变量$z_0$,再通过扩散模型通过$z_0$扩散到$z_1$使得$z_1$服从标准正态分布$z_1\sim \mathcal{N}(z_1;0,I)$。利用逆向微分方程可以从末端$z_1$开始取样通过解方程采样得到$z_0$的分布的近似$p_{\theta}(z_0)$。最后,通过解码器可以通过$p_{\psi}(x|z_0)$将隐式变量映射到原先的样本空间中。在该模型中,和VAE模型的处理方法相同,我们依然通过极大似然估计的方法去进行参数训练。我们在此处最小化对数似然函数相反数的下界(ELBO)来进行参数估计,
\begin{align}
 \mathcal{L}(\mathbf{x}, \boldsymbol{\phi}, \boldsymbol{\theta}, \boldsymbol{\psi})&=\mathbb{E}_{q_{\boldsymbol{\phi}}\left(\mathbf{z}_0 \mid \mathbf{x}\right)}\left[-\log p_{\boldsymbol{\psi}}\left(\mathbf{x} \mid \mathbf{z}_0\right)\right]+\operatorname{KL}\left(q_{\boldsymbol{\phi}}\left(\mathbf{z}_0 \mid \mathbf{x}\right) \| p_{\boldsymbol{\theta}}\left(\mathbf{z}_0\right)\right) \label{Loss decomposition 1}\\
& =\underbrace{\mathbb{E}_{q_{\boldsymbol{\phi}}\left(\mathbf{z}_0 \mid \mathbf{x}\right)}\left[-\log p_{\boldsymbol{\psi}}\left(\mathbf{x} \mid \mathbf{z}_0\right)\right]}_{\text {reconstruction term }}+\underbrace{\mathbb{E}_{q_\phi\left(\mathbf{z}_0 \mid \mathbf{x}\right)}\left[\log q_{\boldsymbol{\phi}}\left(\mathbf{z}_0 \mid \mathbf{x}\right)\right]}_{\text {negative encoder entropy }}+\underbrace{\mathbb{E}_{q_{\boldsymbol{\phi}}\left(\mathbf{z}_0 \mid \mathbf{x}\right)}\left[-\log p_{\boldsymbol{\theta}}\left(\mathbf{z}_0\right)\right]}_{\text {cross entropy }} \label{loss decomposition 2}
\end{align}
通过式(\ref{loss decomposition 2})可以将式(\ref{Loss decomposition 1})中的KL散度进行进一步分解。在式(\ref{loss decomposition 2})中的reconstruction term 和 encoder entropy term都可以通过VAE模型中所使用的重参数化技巧来进行估计优化,在优化中最困难的部分是对交叉熵部分进行估计求导。以下引用在\cite{VAE_diffusion}中得到的关于交叉熵的结论。
\begin{theorem}
在随机微分方程式(\ref{SDE form })下,考虑两个初始分布$q(z_0)$和$p(z_0)$,都定义在$\mathbb{R}^{D}$上,假设$q(z_t)$ ,$p(z_t)$代表在SDE(\ref{SDE form })下在$t\in [0,1]$时刻扩散得到的随机变量的边缘分布。同时假设$\log(p(z_t))$和$\log(q(z_t))$均为光滑函数且在$z_t\to \infty$时增长速度不超过多项式量级增长。同时我们假设所选取的$f(t)$
$g(t)$满足在$t=1$时刻满足$q(z_1)=p(z_1)$,则交叉熵满足以下等式
\begin{align}
    \operatorname{CE}\left(q\left(\mathbf{z}_0 \mid \mathbf{x}\right) \| p\left(\mathbf{z}_0\right)\right) &=\mathbb{E}_{t \sim \mathcal{U}[0,1]}\left[\frac{g(t)^2}{2} \mathbb{E}_{q\left(\mathbf{z}_t, \mathbf{z}_0 \mid \mathbf{x}\right)}\left[\left\|\nabla_{\mathbf{z}_t} \log q\left(\mathbf{z}_t \mid \mathbf{z}_0\right)-\nabla_{\mathbf{z}_t} \log p\left(\mathbf{z}_t\right)\right\|_2^2\right]\right] \nonumber\\
    &+\frac{D}{2} \log \left(2 \pi e \sigma_0^2\right) \label{thm 1}
\end{align}
    其中$q(z_t,z_0|x) = q(z_t|z_0)q(z_0|x)$,以及条件概率分布$q(z_t|z_0)$为正态分布有如下形式$q(z_t|z_0)=\mathcal{N}(z_t;\mu_t(z_0),\sigma_t^2)$。其中$\mu_t$,$\sigma_t^2$由$f(t)$和$g(t)$以及初始时刻$t=0$处方差$\sigma_0^2$唯一确定。
\end{theorem}
根据以上结论,在实际优化过程中需要对$t$时刻的随机变量$z_t$进行采样来进行随机梯度下降,然而根据该种采样得到的随机变量可能有较高的方差,在实际优化中需要采用一定的减小采样方差的方法。 在实际应用中通常会对VPSDE(Variance Preserving SDE)进行分析,即SDE形式为
\begin{equation}
    dz = -\frac{1}{2}\beta(t)zdt + \sqrt{\beta(t)}dW.
    \label{VPSDE}
\end{equation}
其中我们可以定义$\beta(t) = \beta_0+(\beta_1-\beta_0)t$,从而使得该函数取值始终在$[\beta_0,\beta_1]$内。(也可以更换该函数形式满足更加复杂的应用需求)
在\cite{Consistency}中对一致性模型的诸多性质进行了详细的阐述。\par 
在实际应用中,由于图像的维数相对较高,如果直接使用Diffusion Model进行训练(例如使用DDPM等模型)会带来较大的计算资源消耗,因此通常会先将图像通过编码器映射到隐空间中,通过Diffusion Model训练得到隐空间变量的真实分布,最后再通过解码器得到还原后的图像。该方法可以用于图像去噪,图像生成等下游任务。在\cite{High_synthesis}中详细阐述了使用VAE模型和Diffusion Model的结合在图像生成中的丰富应用。以及如果再充分使用Transformer模型,可以进一步实现文字转图像等丰富功能。
\section{条件生成模型}

\begin{figure}[H]
  \centering
  \begin{minipage}[b]{0.3\linewidth}
\includegraphics[width=\linewidth]{Picture/input/00008.png}
    \caption{加噪图像6}
    \label{noised image }
  \end{minipage}
  \hspace{0.1cm} % Space between images
   \begin{minipage}[b]{0.3\linewidth}
    \includegraphics[width=\linewidth]{Picture/label/00008.png}
    \caption{原始图像6}
    \label{original image }
  \end{minipage}
\hspace{0.1cm}
  \begin{minipage}[b]{0.3\linewidth}
    \includegraphics[width=\linewidth]{Picture/recon/00008.png}
    \caption{还原图像6}
    \label{inpainted image}
  \end{minipage}
  \label{整块损坏图像}
\end{figure}
\chapter{针对条件扩散模型的高效图像修复算法}
\section{本章引言}
在\cite{Inverse}中,主要采用如下图像修复算法   

\section{算法设计}
\subsection{后验估计}
通常为了获得后验证估计$p(y\mid x_t)$,我们可以将该条件概率分布写为如下形式
\begin{equation}
    p(y\mid x_t) = \int_{\mathbb{R}^n} p(x_0\mid x_t) p(y\mid x_0,x_t) dx_0,
\end{equation}
我们如果从以下角度来思考本问题,即通过逆向随机微分方程(\ref{reverse SDE conditional})先从$x_t$可以得到$x_0$, 再通过前向加噪模型(\ref{Forward model})得到$y$。根据以上模型的Markov性质可以得到
\begin{equation}
     p(y\mid x_0,x_t) = p(y\mid x_0).
\end{equation}
因此我们可以得到
\begin{equation}
    p(y\mid x_t) = \int_{\mathbb{R}^n} p(x_0\mid x_t) p(y\mid x_0) dx_0=\mathbb{E}\left[p(y\mid x_0)\mid x_t\right],
    \label{conditional posterior formula}
\end{equation}
在\cite{Inverse}中直接采用将$\hat{x}_0=\mathbb{E}\left[x_0\mid x_t\right]$带入\ref{conditional posterior formula}用来逼近真实的后验概率分布,而此逼近方法取决于前向加噪算子$H(\cdot)$的光滑性,以及随着$x$的维数上升其逼近效果也在逐渐下降,因此本文中采用更加精确的后验估计方法进行图像还原。     


为了得到对后验概率分布$p(y\mid x_t)$的逼近,根据\cite{pseudoinverse,ddrm}中的结论,通常可以假设条件可以分布$p(x_0\mid x_t)$可以用$\mathcal{N}(\hat{x}_0(x_t);\gamma_t^2 I)$
来逼近,其中$\gamma_t^2 = \frac{\sigma_t^2}{1+\sigma_t^2}$。则我们可以重新用$x_t$来表示$x_0$。我们可以把$x_0$ 用$\hat{x}_0(x_t) + \gamma \epsilon$ 来逼近,其中我们有$\epsilon\sim \mathcal{N}(0,I)$为服从标准高斯分布的随机变量。 在只考虑$\mathcal{H}$为线性算子的情形下,我们将$y$表示为
\begin{equation}
    y=H(x_0)+n = H(\hat{x}_0+\gamma_t \epsilon)+n = H\hat{x}_0 + (H\epsilon+n).
\end{equation}
根据高斯分布随机变量的可叠加性,$(H\epsilon+n)$仍为高斯分布随机变量,其分布可以表示为$\mathcal{N}(0,\gamma_t^2 H^{\top}H + \sigma_y^2 I)$。因此我们最后可以将$\nabla_{x_t}\log(p(y\mid x_t))$写为
\begin{equation}
  -\frac{1}{2} \nabla_{x_t} \bigg((H\hat{x}_0-y)^{\top}\left(\gamma_t^2 H^{\top}H + \sigma_y^2 I\right)^{-1}(H\hat{x}_0-y)\bigg)
\end{equation}
利用链式法则和线性算子的性质,最后可以将$\nabla_{x_t}\log(p(y\mid x_t))$写为
\begin{equation}
    -\left(\gamma_t^2 H^{\top}H + \sigma_y^2 I\right)^{-1}(H\hat{x}_0-y)\frac{\partial \hat{x}_0(x_t)}{\partial x_t}
\end{equation}

\subsection{更新迭代过程}
在\cite{Inverse}中使用如下算法来进行更新迭代,最后得到$\hat{x}_0$作为对原始图像的逼近。在每一步更新迭代中,采用两步进行。第一步,先利用预训练集得到在原始无条件分布下的score function,从而得到在无条件生成通过DDPM采样更新得到的$x_{i-1}^{\prime}$以及得到$\hat{x}_0$。 第二步, 得到通过$\nabla_{x_i}\log(y\mid \hat{x}_0)$来更新$x_{i-1}^{\prime}$得到$x_{i-1}$。具体算法流程图如下算法\ref{DPS algorithm }可见
\begin{breakablealgorithm}
\caption{DPS Algorithm }
\label{DPS algorithm }
    \begin{algorithmic}[1]
   \REQUIRE Input $N$, $y$, $\{\zeta_i\}_{i=1}^{N}$, $\{\Tilde{\sigma}_i\}_{i=1}^{T}$
   \STATE $X_{T}\sim \mathcal{N}(0,\boldsymbol{I})$
   \FOR{$i$ = $N$ to 0}
   \STATE Set $\hat{s}\xleftarrow{} s_{\theta}(x_i,i)$
   \STATE Set $\hat{x_0} \xleftarrow{} \frac{1}{\sqrt{\Bar{\alpha}_i}}\left(x_i+(1-\Bar{\alpha}_i)\hat{s}\right)$
   \STATE Simulate $z\sim \mathcal{N}(0,\boldsymbol{I})$
   \STATE Set $x^{\prime}_{i-1}\xleftarrow{} \frac{\sqrt{\alpha_i}\left(1-\bar{\alpha}_{i-1}\right)}{1-\bar{\alpha}_i} \boldsymbol{x}_i+\frac{\sqrt{\bar{\alpha}_{i-1}} \beta_i}{1-\bar{\alpha}_i} \hat{\boldsymbol{x}}_0+\Tilde{\sigma}_i {z} $
   \STATE Update $x_{i-1}\xleftarrow{} {x}_{i-1}^{\prime}-\zeta_i \nabla_{{x}_t}\left\|{y}-\mathcal{A}\left(\hat{{x}}_0\right)\right\|_2^2 $
   \ENDFOR
   \RETURN $\hat{x}_0$ 
    \end{algorithmic}
\end{breakablealgorithm}      

两步迭代主要利用了在条件生成下的逆向随机微分方程的结构
\begin{equation}
    dx = \left(f(x,t) - g^2(t)(\nabla_{x_t}\log(p(x_t)))+ \nabla_{x_t}\log(y\mid x(t))\right)+g(t)d\Tilde{w}.
\end{equation}
第一步先更新除了带$\nabla_{x_t}\log(y\mid x(t))$的项,接下来再单独更新$\nabla_{x_t}\log(y\mid x(t))$的项。理论上$\nabla_{x_t}\log(y\mid x(t))$可以写为$-\frac{1}{\sigma_y^2}\nabla_{x_t}\|y-H(\hat{x}_0(x_t))\|_2^2$的形式,但是为了保证稳定性常常需要再前面乘以一个步长$\zeta_t$来保证收敛性,而在\cite{Inverse}中该步长为根据数据集本身形式人工选取,具有不稳定性和一定的随机性,不一定为最优的参数选取。如果$\sigma_y$过大则会导致每次更新步长多大如果$\sigma_y$过小则会导致每次更新步长不明显,最终生成的图片不一定能够对应损坏图片。因此在本文中采取如下方法,我们首先对应每次更新$\nabla_{x_t}\log(y\mid x(t))$项的时候的步长设置为$\eta_i$, 以及我们从$x_{i-1}^{\prime}$生成$x_{i-1}$的时候采用如下生成方式
\begin{equation}
    x_{i-1} \xleftarrow{} x_{i-1}^{\prime} - \eta_i \nabla_{x_i}\sqrt{((H\hat{x}_0-y)^{\top}\left(\gamma_i^2 H^{\top}H + \sigma_y^2 I\right)^{-1}(H\hat{x}_0-y)\bigg)}.
\end{equation}
该方法的优势在于不需要对步长刻意设置,我们在本模型中直接取$\eta_i=1$, $i=1,2,\cdots,T$, 也具有较好的稳定性。 如果直接取所有$\gamma_i=0$, $i=1,2,\cdots, T$, 则该更新方法可以写为
\begin{equation}
       x_{i-1} \xleftarrow{} x_{i-1}^{\prime} - \eta_i \nabla_{x_i}\|y-H(\hat{x}_0)\|_2,
\end{equation}
对右式继续进行化简可得
\begin{align}
    x_{i-1}^{\prime} - \eta_i \nabla_{x_i}\|y-H(\hat{x}_0)\|_2 &=  x_{i-1}^{\prime} - \frac{\eta_i}{\|y-H(\hat{x}_0)\|_2} H^{\top}(H\hat{x}_0-y)\frac{\partial \hat{x}_0}{\partial x_i}\\
    & = x_{i-1}^{\prime} - {\eta_i} H^{\top}\frac{(H\hat{x}_0-y)}{\|H\hat{x}_0-y\|_2}\frac{\partial \hat{x}_0}{\partial x_i},
\end{align}
中间的第二项里面的$\frac{(H\hat{x}_0-y)}{\|H\hat{x}_0-y\|_2}$已经被标注化,从而$\eta_i$可以控制每次更新的幅度,在原来的算法中如果仅仅对$\nabla_{x_i}\|H\hat{x}_0-y\|_2^2$进行步长选取,当$\|H\hat{x}_0-y\|$ 已经足够小的时候则不再更新,可能会陷入局部最优的解。(例如最后得到的还原图像确实经过加噪模型可以得到损坏图像,但是与原图像有较大的区别)经过实验该更新方法也确实有更好的鲁棒性。 

\subsection{算法流程}
基于DPS算法,我们现在提出我们采用更高效率基于DDIM模型的算法。在本文的改进算法中,可以采用DDPM模型进行采样,也可以对DDPM的模型中的总步长进行分割进行切片采样。进一步,还可以利用DDIM方法来进行采样,原先使用DDPM采样方法需要进行1000步,在DDIM方法下只需要100步就可以获得质量相似的图像。本文所提出的算法流程图如下所示

\begin{breakablealgorithm}
\caption{Image Restoration Algorithm }
\label{ours algorithm }
    \begin{algorithmic}[1]
   \REQUIRE Input $T$, $y$, $\{\eta_k\}_{k=1}^{T}$, $\{\sigma_k\}_{k=1}^{T}$
   \REQUIRE Choose $\{\tau_k\}_{k=1}^{m}$ according to the sampling scheme
   \STATE $X_{T}\sim \mathcal{N}(0,\boldsymbol{I})$
   \FOR{$t=m$, $m-1$, $\cdots$, $0$}
   \STATE Set $\hat{s}\xleftarrow{} s_{\theta}(\tau_t,x_{\tau_t})$
   \STATE Set $\hat{x_0} \xleftarrow{} \frac{1}{\sqrt{\Bar{\alpha}_{\tau_t}}}\left(x_{\tau_t}+\Bar{\beta}_{\tau_t}\hat{s}\right)$
   \STATE Simulate $z\sim \mathcal{N}(0,\boldsymbol{I})$
   \STATE Set $x^{\prime}_{\tau_{t-1}}\xleftarrow{} \frac{\sqrt{\alpha_{\tau_t}}\left(1-\bar{\alpha}_{\tau_{t-1}}\right)}{1-\bar{\alpha}_{\tau_t}} \boldsymbol{x}_{\tau_t}+\frac{\sqrt{\bar{\alpha}_{\tau_{t-1}}} \beta_{\tau_t}}{1-\bar{\alpha}_{\tau_t}} \hat{\boldsymbol{x}}_0+\sigma_{y} {z} $
   \STATE Update $x_{\tau_{t-1}}\xleftarrow{} {x}_{\tau_{t-1}}^{\prime}-\eta_t \nabla_{x_{\tau_t}}\sqrt{((H\hat{x}_0-y)^{\top}\left(\gamma_{\tau_t}^2 H^{\top}H + \sigma_y^2 I\right)^{-1}(H\hat{x}_0-y)\bigg)}. $
   \ENDFOR
   \RETURN $\hat{x}_0$ 
    \end{algorithmic}
\end{breakablealgorithm}


\section{参数选取与模型设计}
为了实现加噪过程以及去噪过程,需要以下结构: 
\begin{itemize}
    \item 加噪模型,其中包括前向算子$H$和对于噪声$n$的刻画,在该实验中,我们主要对低分辨算子,图像损坏算子来进行实验, 同时我们仅考虑$n$服从高斯分布的情形,即$n\sim \mathcal{N}(0,\sigma^2 I)$,其中我们取$\sigma=0.05$。 
    \item Diffusion model设置, 对于任意数据集,我们均选取 \href{https://github.com/openai/guided-diffusion}{https://github.com/openai/guided-diffusion} 中的各个数据集的标准预训练集来加载预训练集,对于具体的扩散模型使用说明可见附录 \ref{appendix}。 
    \item  后验估计,即在已知$z_t$, $y$ 的情形下,给出对于$\nabla_{z_t}\log\left(q(y\mid z_t)\right)$的估计。 在本文实验中利用算法\ref{ours algorithm }进行后验证估计。
    \item 条件采样过程,即在即在已知$z_t$, $y$ 的情形下,给出采样得到$z_{t-1}$的过程。 在算法\ref{ours algorithm }中,不失一般性,我们取$\eta_i=1.0$, 该参数不需要刻意微调,一般设置在0.5-1.0均可以实现较好的图像生成效果。 
    \item 下游任务设置,在本项目中主要选取图像损坏 
  (Image Inpainting)、 超分辨率 (Super Resolution) 、 图像线性去模糊化 (Motion Deblurring) 和 图像非线性去模糊化 (Nonlinear Deblurring) 下游任务进行实验。其中对于线性去模糊化和非线性去模糊化下游任务中本文利用了github仓库 \href{https://github.com/VinAIResearch/blur-kernel-space-exploring}{https://github.com/VinAIResearch/blur-kernel-space-exploring} 和 \href{https://github.com/LeviBorodenko/motionblur}{https://github.com/LeviBorodenko/motionblur} 的实现方法。 
\end{itemize}



\section{实验结果}
首先,本文对FFHQ数据集进行测试,我们以图像修复任务 (Image Inpainting)进行可视化展现实验结果。在实验中我们首先采用随机加噪方法进行前向加噪,对于每个像素有30\%的概率损坏变成白噪声,如下图\ref{noised image 1}即为加噪后的图像。 
\begin{figure}[H]
  \centering
  \begin{minipage}[b]{0.3\linewidth}
\includegraphics[width=\linewidth]{Picture/input/00000.png}
    \caption{加噪图像1}
    \label{noised image 1}
  \end{minipage}
  \hspace{0.1cm} % Space between images
   \begin{minipage}[b]{0.3\linewidth}
    \includegraphics[width=\linewidth]{Picture/label/00000.png}
    \caption{原始图像1}
    \label{original image 1 }
  \end{minipage}
\hspace{0.1cm}
  \begin{minipage}[b]{0.3\linewidth}
    \includegraphics[width=\linewidth]{Picture/recon/00000.png}
    \caption{还原图像1}
    \label{inpainted image 1}
  \end{minipage}
\end{figure}

\begin{figure}[H]
  \centering
  \begin{minipage}[b]{0.3\linewidth}
\includegraphics[width=\linewidth]{Picture/input/00001.png}
    \caption{加噪图像2}
    \label{noised image 2}
  \end{minipage}
  \hspace{0.1cm} % Space between images
   \begin{minipage}[b]{0.3\linewidth}
    \includegraphics[width=\linewidth]{Picture/label/00001.png}
    \caption{原始图像2}
    \label{original image 2}
  \end{minipage}
\hspace{0.1cm}
  \begin{minipage}[b]{0.3\linewidth}
    \includegraphics[width=\linewidth]{Picture/recon/00001.png}
    \caption{还原图像2}
    \label{inpainted image 2}
  \end{minipage}
\end{figure}


\begin{figure}[H]
  \centering
  \begin{minipage}[b]{0.3\linewidth}
\includegraphics[width=\linewidth]{Picture/input/00002.png}
    \caption{加噪图像3}
    \label{noised image 3}
  \end{minipage}
  \hspace{0.1cm} % Space between images
   \begin{minipage}[b]{0.3\linewidth}
    \includegraphics[width=\linewidth]{Picture/label/00002.png}
    \caption{原始图像3}
    \label{original image 3}
  \end{minipage}
\hspace{0.1cm}
  \begin{minipage}[b]{0.3\linewidth}
    \includegraphics[width=\linewidth]{Picture/recon/00002.png}
    \caption{还原图像3}
    \label{inpainted image 3}
  \end{minipage}
\end{figure}
以及如下图\ref{noised image 4}即为整块损坏图像,本文同时给出图像修复的还原过程切片。 更多实验结果可见附录\ref{appendix}。 



\begin{figure}[H]
  \centering
  \begin{minipage}[b]{0.3\linewidth}
\includegraphics[width=\linewidth]{Picture/input/input_box.png}
    \caption{加噪图像4}
    \label{noised image 4}
  \end{minipage}
  \hspace{0.1cm} % Space between images
   \begin{minipage}[b]{0.3\linewidth}
    \includegraphics[width=\linewidth]{Picture/label/label_box.png}
    \caption{原始图像4}
    \label{original image 4}
  \end{minipage}
\hspace{0.1cm}
  \begin{minipage}[b]{0.3\linewidth}
    \includegraphics[width=\linewidth]{Picture/recon/recon_box.png}
    \caption{还原图像4}
    \label{inpainted image 4}
  \end{minipage}
  \label{整块损坏图像}
\end{figure}


\begin{figure}[H]
  \centering
  \begin{minipage}[b]{0.3\linewidth}
\includegraphics[width=\linewidth]{Picture/progress/box/x_0500.png}
  \end{minipage}
  \hspace{0.1cm} % Space between images
   \begin{minipage}[b]{0.3\linewidth}
    \includegraphics[width=\linewidth]{Picture/progress/box/x_0400.png}
  \end{minipage}
\hspace{0.1cm}
  \begin{minipage}[b]{0.3\linewidth}
    \includegraphics[width=\linewidth]{Picture/progress/box/x_0200.png}
  \end{minipage}
\end{figure}

\begin{figure}[H]
  \centering
  \begin{minipage}[b]{0.3\linewidth}
\includegraphics[width=\linewidth]{Picture/progress/box/x_0500.png}
  \end{minipage}
  \hspace{0.1cm} % Space between images
   \begin{minipage}[b]{0.3\linewidth}
    \includegraphics[width=\linewidth]{Picture/progress/box/x_0100.png}
  \end{minipage}
\hspace{0.1cm}
  \begin{minipage}[b]{0.3\linewidth}
    \includegraphics[width=\linewidth]{Picture/progress/box/x_0000.png}
  \end{minipage}
\end{figure}




以下为在PSNR和SSIM两个度量下在不同图像修复算法下的表现,由此可以看到总体而言本文算法优于大部分图像修复算法,但是相比于在\cite{ddrm}中提出的DDRM算法仍然还有改善空间。 在图像修复任务(即Inpainting任务)中,本文所提出的算法相比于DPS算法有更好的表现,已经可以还原得到质量较高的图像。然而在整块损坏的图像修复任务中仍然不如DDRM\cite{ddrm}中所提出的算法,因为在整块挖去的情形下,对于后验分布估计相比于随机加噪的情形下无法利用损坏图像区域周围的像素点对图像进行充分还原。 
\begin{table}[H]
    \centering
    \begin{tabular}{|c|c|c|c|c|}
\hline \multirow[b]{2}{*}{ Method } & \multicolumn{2}{|c|}{ Inpaint (random)} & \multicolumn{2}{|c|}{ Inpaint (box) } \\
\hline & PSNR $\uparrow$ & $\operatorname{SSIM} \uparrow$ & PSNR $\uparrow$ & $\operatorname{SSIM} \uparrow$ \\
\hline \text{本文算法} & ${24.12}$ & $\underline{0.813}$ & \underline{18.95} & ${0.797}$ \\
\hline DPS\cite{Inverse}  & ${23.87}$ & ${0.781}$ & {18.90} & ${0.794}$ \\
\hline DDRM \cite{ddrm} & \underline{24.96} & 0.790 & ${18.66}$ & \underline{0.814} \\
\hline MCG\cite{MCG}  & 13.39 & 0.227 & 17.36 & 0.633 \\
\hline PnP-ADMM\cite{PnP}  & 23.75 & 0.761 & 12.70 & 0.657 \\
\hline \begin{tabular}{l} 
Score-SDE \cite{score_based_SDE} \\
\end{tabular} & 12.25 & 0.256 & 16.48 & 0.612 \\
\hline ADMM-TV & 22.17 & 0.679 & 17.96 & 0.785 \\
\hline
\end{tabular}
\caption{不同算法下的图像去噪结果}
\end{table}
以上为使用FFHQ数据集下进行图像修复的实验结果,总体而言在进行随机图像损坏任务中,在本文的算法改进下,在PSNR度量下优于DPS算法的图像修复结果,但是效果不如DDRM算法。但是考虑到DDRM算法在每一次反向迭代过程中需要每次对加噪算子$H$进行奇异值分解并且对图像逐元素进行更新, 因此在时间复杂度上不如本文提出的算法,以及在SSIM度量下本文的算法优于DDRM算法。 以及在对图像进行整块挖出的图像修复任务中,本文提出的算法在PSNR富相俩优于DPS算法,而在SSIM度量下仍然不如DDRM算法,这是由该任务的特性决定的。在对图像进行整块挖除的情形下,由于DDRM算法在对前向加噪算子进行起奇异值分解后,在逆向过程进行反向传播的过程中可以保留原图像的未损失图像的全部信息,对于整块挖除的损坏图像可以直接保留原始未损失图像。因此在DDRM算法下只需要对损坏区域单独进行修复即可,大大增加了图像修复的处理效率。 而在本文算法中,每一次迭代过程中将图像整体作为输入统一进行修复,而不对损坏区域进行特殊处理,因此相比而言在该任务下存在劣势。    


相比于在\cite{Inverse}中仅仅对FFHQ数据集进行测试,本文选择LSUN数据集中的Bedrooms卧室图像数据集进行测试,同样利用两种图像加噪方式 (随机加噪于整块挖除)进行实验,如下图为实验结果。此处我们选取了不同图像作为例子展现图像修复的还原过程。我们在图例中放了三张图像作为对比,最左边的是加噪后的图像,最中间的是原始图像作为ground truth和还原后的图像进行对比分析,最右边的图像为经过本文图像去噪算法后还原得到后的图像。 
\begin{figure}[H]
  \centering
  \begin{minipage}[b]{0.3\linewidth}
\includegraphics[width=\linewidth]{Picture/input/00007.png}
    \caption{加噪图像5}
    \label{noised image 5 }
  \end{minipage}
  \hspace{0.1cm} % Space between images
   \begin{minipage}[b]{0.3\linewidth}
    \includegraphics[width=\linewidth]{Picture/label/00007.png}
    \caption{原始图像5}
    \label{original image  5}
  \end{minipage}
\hspace{0.1cm}
  \begin{minipage}[b]{0.3\linewidth}
    \includegraphics[width=\linewidth]{Picture/recon/00007.png}
    \caption{还原图像5}
    \label{inpainted image 5}
  \end{minipage}
  \label{整块损坏图像}
\end{figure}

\begin{figure}[H]
  \centering
  \begin{minipage}[b]{0.3\linewidth}
\includegraphics[width=\linewidth]{Picture/input/00008.png}
    \caption{加噪图像6}
    \label{noised image 6 }
  \end{minipage}
  \hspace{0.1cm} % Space between images
   \begin{minipage}[b]{0.3\linewidth}
    \includegraphics[width=\linewidth]{Picture/label/00008.png}
    \caption{原始图像6}
    \label{original image 6}
  \end{minipage}
\hspace{0.1cm}
  \begin{minipage}[b]{0.3\linewidth}
    \includegraphics[width=\linewidth]{Picture/recon/00008.png}
    \caption{还原图像6}
    \label{inpainted image 6}
  \end{minipage}
  \label{整块损坏图像}
\end{figure}


\begin{figure}[H]
  \centering
  \begin{minipage}[b]{0.3\linewidth}
\includegraphics[width=\linewidth]{Picture/input/00009.png}
    \caption{加噪图像7}
    \label{noised image 7 }
  \end{minipage}
  \hspace{0.1cm} % Space between images
   \begin{minipage}[b]{0.3\linewidth}
    \includegraphics[width=\linewidth]{Picture/label/00009.png}
    \caption{原始图像7}
    \label{original image 7 }
  \end{minipage}
\hspace{0.1cm}
  \begin{minipage}[b]{0.3\linewidth}
    \includegraphics[width=\linewidth]{Picture/recon/00009.png}
    \caption{还原图像7}
    \label{inpainted image 7}
  \end{minipage}
  \label{整块损坏图像}
\end{figure}



\begin{figure}[H]
  \centering
  \begin{minipage}[b]{0.3\linewidth}
\includegraphics[width=\linewidth]{Picture/input/00010.png}
    \caption{加噪图像8}
    \label{noised image  8}
  \end{minipage}
  \hspace{0.1cm} % Space between images
   \begin{minipage}[b]{0.3\linewidth}
    \includegraphics[width=\linewidth]{Picture/label/00010.png}
    \caption{原始图像8}
    \label{original image 8 }
  \end{minipage}
\hspace{0.1cm}
  \begin{minipage}[b]{0.3\linewidth}
    \includegraphics[width=\linewidth]{Picture/recon/00010.png}
    \caption{还原图像8}
    \label{inpainted image 8}
  \end{minipage}
  \label{整块损坏图像}
\end{figure}

\begin{figure}[H]
  \centering
  \begin{minipage}[b]{0.3\linewidth}
\includegraphics[width=\linewidth]{Picture/progress/random/x_0500.png}
  \end{minipage}
  \hspace{0.1cm} % Space between images
   \begin{minipage}[b]{0.3\linewidth}
    \includegraphics[width=\linewidth]{Picture/progress/random/x_0400.png}
  \end{minipage}
\hspace{0.1cm}
  \begin{minipage}[b]{0.3\linewidth}
    \includegraphics[width=\linewidth]{Picture/progress/random/x_0200.png}
  \end{minipage}
\end{figure}

\begin{figure}[H]
  \centering
  \begin{minipage}[b]{0.3\linewidth}
\includegraphics[width=\linewidth]{Picture/progress/random/x_0500.png}
  \end{minipage}
  \hspace{0.1cm} % Space between images
   \begin{minipage}[b]{0.3\linewidth}
    \includegraphics[width=\linewidth]{Picture/progress/random/x_0100.png}
  \end{minipage}
\hspace{0.1cm}
  \begin{minipage}[b]{0.3\linewidth}
    \includegraphics[width=\linewidth]{Picture/progress/random/x_0000.png}
  \end{minipage}
\end{figure}

我们在FFHQ数据集上同时进行了更加丰富的实验,我们对不同图像均进行了图像修复操作,结果展示如下。 以下为在FFHQ数据集上测试的结果展示
\begin{figure}[H]
  \centering
  \begin{minipage}[b]{0.3\linewidth}
\includegraphics[width=\linewidth]{Picture/input/00004.png}
    \caption{加噪图像9}
    \label{noised image }
  \end{minipage}
  \hspace{0.1cm} % Space between images
   \begin{minipage}[b]{0.3\linewidth}
    \includegraphics[width=\linewidth]{Picture/label/00004.png}
    \caption{原始图像9}
    \label{original image }
  \end{minipage}
\hspace{0.1cm}
  \begin{minipage}[b]{0.3\linewidth}
    \includegraphics[width=\linewidth]{Picture/recon/00004.png}
    \caption{还原图像9}
    \label{inpainted image}
  \end{minipage}
\end{figure}



\begin{figure}[H]
  \centering
  \begin{minipage}[b]{0.3\linewidth}
\includegraphics[width=\linewidth]{Picture/input/00005.png}
    \caption{加噪图像10}
    \label{noised image }
  \end{minipage}
  \hspace{0.1cm} % Space between images
   \begin{minipage}[b]{0.3\linewidth}
    \includegraphics[width=\linewidth]{Picture/label/00005.png}
    \caption{原始图像10}
    \label{original image }
  \end{minipage}
\hspace{0.1cm}
  \begin{minipage}[b]{0.3\linewidth}
    \includegraphics[width=\linewidth]{Picture/recon/00005.png}
    \caption{还原图像10}
    \label{inpainted image}
  \end{minipage}
\end{figure}

\begin{figure}[H]
  \centering
  \begin{minipage}[b]{0.3\linewidth}
\includegraphics[width=\linewidth]{Picture/input/00006.png}
    \caption{加噪图像11}
    \label{noised image }
  \end{minipage}
  \hspace{0.1cm} % Space between images
   \begin{minipage}[b]{0.3\linewidth}
    \includegraphics[width=\linewidth]{Picture/label/00006.png}
    \caption{原始图像11}
    \label{original image }
  \end{minipage}
\hspace{0.1cm}
  \begin{minipage}[b]{0.3\linewidth}
    \includegraphics[width=\linewidth]{Picture/recon/00006.png}
    \caption{还原图像11}
    \label{inpainted image}
  \end{minipage}
\end{figure}


\section{更多结果}
本文所精心设计的图像还原算法展现出了卓越的泛化能力,这一能力使其能够轻松应对不同数据集下的图像还原任务。不论面对的是何种类型的图像数据,该算法都能够凭借其高效的运算机制,实现精准的图像还原效果。值得一提的是,这种算法的独特之处在于其对超参数的宽容性。在实际应用中,我们无需对超参数进行繁琐且精细的调整,算法便能够自动适应不同的图像特性,从而确保还原效果的稳定性和可靠性。    


为了进一步验证这一算法的实际效果,我们特别选取了两位国际一线篮球巨星的图像进行实验。这两位篮球巨星不仅在全球范围内拥有极高的知名度和影响力,他们的形象也因其独特的个人风格和比赛风采而深入人心。通过将这些篮球巨星的图像作为实验样本,我们能够更加直观地感受到该算法在图像还原方面的卓越性能。实验结果表明,无论是对于清晰度、色彩还原度还是细节表现,该算法都展现出了令人满意的效果,充分证明了其在实际应用中的价值和潜力。
\begin{figure}[H]
  \centering
  \begin{minipage}[b]{0.3\linewidth}
\includegraphics[width=\linewidth]{Picture/input/kun1_input.png}
  \end{minipage}
  \hspace{0.1cm} % Space between images
   \begin{minipage}[b]{0.3\linewidth}
    \includegraphics[width=\linewidth]{Picture/label/kun1_label.png}
  \end{minipage}
\hspace{0.1cm}
  \begin{minipage}[b]{0.3\linewidth}
    \includegraphics[width=\linewidth]{Picture/recon/kun1_recon.png}
  \end{minipage}
  \caption{球星图像修复1}
\end{figure}



\begin{figure}[H]
  \centering
  \begin{minipage}[b]{0.3\linewidth}
\includegraphics[width=\linewidth]{Picture/input/kun2_input.png}
  \end{minipage}
  \hspace{0.1cm} % Space between images
   \begin{minipage}[b]{0.3\linewidth}
    \includegraphics[width=\linewidth]{Picture/label/kun2_label.png}
  \end{minipage}
\hspace{0.1cm}
  \begin{minipage}[b]{0.3\linewidth}
    \includegraphics[width=\linewidth]{Picture/recon/kun2_recon.png}
  \end{minipage}
  \caption{球星图像修复2}
\end{figure}


\begin{figure}[H]
  \centering
  \begin{minipage}[b]{0.3\linewidth}
\includegraphics[width=\linewidth]{Picture/input/mamba_input.png}
  \end{minipage}
  \hspace{0.1cm} % Space between images
   \begin{minipage}[b]{0.3\linewidth}
    \includegraphics[width=\linewidth]{Picture/label/mamba_label.png}
  \end{minipage}
\hspace{0.1cm}
  \begin{minipage}[b]{0.3\linewidth}
    \includegraphics[width=\linewidth]{Picture/recon/mamba_recon.png}
  \end{minipage}
  \caption{球星图像修复3}
\end{figure}




\section{本章小结}

\chapter{总结与期望}
\section{本文总结}
本论文深入探讨了图像修复领域的核心问题,特别是针对因加噪算子影响而受损的图像,我们精心构建并提出了一种卓越的图像修复算法——\ref{ours algorithm }。该算法不仅是对\cite{Inverse}中后验估计思想的继承,更是对其进行了深入的拓展和革新。结合加噪算子的独特性质以及高斯分布的理论基础,我们设计了一种更为精确且高效的后验估计算法,从而实现了对受损图像更为精准的修复。

在算法的具体实现上,我们巧妙地融合了DDPM采样方法,并结合\cite{DDIM}中提出的分段采样策略。这种结合不仅极大地提升了图像生成的效率,更确保了图像质量的显著提升。因此,在处理大规模或高分辨率图像时,我们的算法能够展现出卓越的性能,为图像修复领域带来了新的技术突破。

值得一提的是,我们成功解决了\cite{Inverse}中存在的迭代学习率调整问题。通过精心设计和优化算法结构,我们实现了算法在不同数据集下以稳定的学习率进行图像还原,这极大地增强了算法的通用性和实用性,使得我们的算法能够在更广泛的场景中发挥出色的性能。

为了验证算法的有效性和稳定性,我们在LSUN和FFHQ等多个数据集上进行了全面的实验测试。实验结果表明,我们的算法在这些数据集上均取得了令人满意的图像生成效果,充分证明了其高效性和稳定性。这些实验结果不仅展示了算法在不同数据集下的广泛应用前景,更为我们未来的研究提供了有力的支撑。

综上所述,本论文提出的\ref{ours algorithm }算法在图像修复领域取得了显著的成果,为解决加噪算子作用下的图像损坏问题提供了新的思路和方法。展望未来,我们将继续致力于技术创新和算法优化,以期在图像修复领域取得更多的突破和进展,为图像处理技术的发展做出更大的贡献。
\section{未来工作展望}
在未来,我们将持续投入精力,致力于提升算法的效率和性能,以推动图像修复技术的革新与发展。我们的研究将从以下几个维度深入展开:

首先,为了拓展算法的适用范围,我们将深入剖析不同类型的加噪算子。通过对其特性的细致研究,我们期望能够设计出更为精准和高效的后验估计算法,以应对更多样化的图像损坏问题。这将使我们的算法在面对各种复杂的噪声和损坏情况时,都能展现出卓越的修复能力。

其次,我们将持续优化算法的结构和参数设置。通过引入先进的优化算法和自适应学习率调整机制,我们期望能够显著提升算法的收敛速度和稳定性。这不仅可以使算法在训练过程中更加高效,减少人工调整参数的繁琐,更能确保算法在不同场景下的稳定表现。同时,在逆向采样的效率上,我们也将寻求新的突破,例如结合\cite{Consistency}中的研究成果,以加速从噪声映射到目标图像的转换过程,进一步提高图像修复的速度和精度。

此外,我们还将积极探索将本算法与其他图像生成和修复技术相结合的可能性。我们相信,通过与深度学习中的生成对抗网络(GAN)等技术的融合,我们可以进一步提升图像修复的质量和多样性。同时,我们也将保持对图像修复领域内外最新技术进展的敏锐洞察,不断尝试将新技术引入到我们的算法中,以持续提高算法的性能和实用性。

最后,我们将继续开展广泛的实验验证工作。我们将收集更多的数据集,并在其上对本算法进行测试。通过收集和分析实验结果,我们将更全面地评估算法的性能,并发现其中可能存在的问题和不足。这将为我们未来的研究工作提供宝贵的参考和指导,帮助我们不断优化算法,提高其在现实世界中的应用价值。

在\cite{Inverse}和\cite{song2023pseudoinverse}的研究中,虽然算法在特定数据集上取得了良好的验证效果,但其在更多数据集上的泛化能力仍有待加强。因此,我们将致力于开发泛化能力更强的图像修复算法,以应对现实世界中更加复杂和多样化的图像修复需求。

综上所述,我们坚信,通过不断的研究和创新,我们将能够在图像修复领域取得更多的突破和进展,为图像处理技术的发展贡献自己的力量。
\backmatter

% 参考文献
\bibliography{ref/refs}  % 
\appendix
% !TeX root = ../thuthesis-example.tex

\begin{survey}
\label{cha:survey}

\title{Literature Review}
\maketitle


% \tableofcontents

The field of image restoration has witnessed a paradigm shift with the advent of deep learning technologies. Among these, diffusion models have emerged as particularly potent, owing to their ability to reconstruct high-quality images from corrupted inputs. This review delves into the evolution and contributions of diffusion models, especially focusing on Score-Based Stochastic Differential Equations \cite{score_based_SDE}, Denoising Diffusion Probabilistic Models \cite{DDPM}, and Denoising Diffusion Implicit Models \cite{DDIM}, while also discussing the role of Variational Autoencoders \cite{VAE_diffusion} and the integration of manifold constraint learning \cite{MCG}.  VAEs, as introduced by \cite{vae_model}, marked a significant milestone in image restoration, providing a framework for modeling complex image distributions. Despite their contributions, the quest for models that better capture high-dimensional data distributions led to the exploration of diffusion models, which offer a novel approach to image restoration.       

Diffusion models, particularly DDPM, introduced by \cite{DDPM}, have redefined image restoration strategies. These models employ a reverse diffusion process to generate data from noise, demonstrating unparalleled performance in handling complex image degradations.    


The incorporation of Score-Based SDEs into diffusion models, as explored by \cite{score_based_SDE,song_2}, has further enhanced the field. These models use the data distribution's score-based gradient for a more refined denoising process, leading to superior restoration quality.    

DDIM, a notable advancement over DDPM, was introduced by \cite{DDIM} to improve efficiency in the denoising process. This model significantly accelerates image restoration without compromising output quality, highlighting its potential for real-time applications.   

Manifold constraint learning has been pivotal in ensuring that restored images adhere to natural image characteristics. The integration of this concept with diffusion models, especially in works like \cite{Inverse,PnP,MCG}, demonstrates significant improvements in photorealism and structural coherence.      


Recent literature has expanded the scope of diffusion models in image restoration. The DDRM framework, as discussed by \cite{ddrm}, and the Pseudo Inverse Algorithm \cite{red_diff} offer novel approaches to tackling image restoration tasks, emphasizing the versatility and depth of diffusion-based methodologies. Furthermore, the emergence of consistency models \cite{Consistency} and Stable Diffusion  \cite{vae_model} highlight the ongoing innovation within the domain, promising even more sophisticated solutions to image restoration challenges.       


The exploration of diffusion models in image restoration has ushered in a new era of possibilities, with DDPM, DDIM, and the integration of Score-Based SDEs representing significant advancements. The addition of manifold constraint learning and the exploration of new models like DDRM and the Pseudo Inverse Algorithm further enrich the field. As research continues to evolve, the foundational works and recent innovations collectively point towards a future where image restoration achieves unprecedented levels of accuracy and realism.



\bibliographystyle{unsrtnat}
\bibliography{ref/refs}

\end{survey}
       % 本科生:外文资料的调研阅读报告
% \input{data/appendix-translation}  % 本科生:外文资料的书面翻译
% !TeX root = ../thuthesis-example.tex

\chapter{补充内容}
\label{appendix}
\section{实验说明}
本项目为开源项目,可以直接从github上获取本项目代码,在终端输入如下命令行。1. 拷贝项目
\label{experiment instructions}
\begin{lstlisting}[language=bash]
git clone git@github.com:JamesJunyuGuo/Graduation_Project.git
cd Graduation_Project/
\end{lstlisting}
2. 下载预训练集。 参考github仓库\href{https://github.com/openai/guided-diffusion.git}{https://github.com/openai/guided-diffusion.git}可以点击进入网页下载对应的预训练集。在主文件夹中新建一个新文件夹 models/ 并将下载的预训练集放到里面。
\begin{lstlisting}[language=bash]
mkdir models
mv {DOWNLOAD_DIR}/ffqh_10m.pt ./models/
mv {DOWNLOAD_DIR}/lsun_bedroom.pt ./models/
\end{lstlisting}
3. 本地环境配置
\begin{lstlisting}[language=bash]
conda create --name DM python=3.8

conda activate DM

pip install -r requirements.txt

pip install torch==2.2.2
\end{lstlisting}
4. 训练过程   
执行训练只需要在本地命令行输入如下代码
\begin{lstlisting}[language=bash]
python3 sample_test.py \
--model_config=configs/LSUN_config.yaml \
--diffusion_config=configs/diffusion_config.yaml \
--task_config=inpainting_config.yaml;
\end{lstlisting}
其中可以根据具体的下游任务类型修改yaml文件中的具体配置,例如在任务设置参数中可以修改root以及name更改读取数据集位置以为选取不同的下游任务进行训练,文件配置内容如下可见。
\begin{lstlisting}[language=bash]
conditioning:
    method: ddim 
    params:
        scale: 1.0

data:
    name: LSUN
    root: ./data/samples/

measurement:
    operator: 
        name: inpainting # check candidates in guided_diffusion/measurements.py

noise:
    name:  gaussian 
    sigma:  0.05 
\end{lstlisting}








% 致谢
% !TeX root = ../thuthesis-example.tex

\begin{acknowledgements}
  衷心感谢导师包承龙教授和邹易航对本人的精心指导。他们的言传身教将使我终生受益。

  
\end{acknowledgements}


% 声明
\statement
% 将签字扫描后的声明文件 scan-statement.pdf 替换原始页面
% \statement[file=scan-statement.pdf]
% 本科生编译生成的声明页默认不加页脚,插入扫描版时再补上;
% 研究生编译生成时有页眉页脚,插入扫描版时不再重复。
% 也可以手动控制是否加页眉页脚
% \statement[page-style=empty]
% \statement[file=scan-statement.pdf, page-style=plain]



% 本科生的综合论文训练记录表(扫描版)
% \record{file=scan-record.pdf}

\end{document}
