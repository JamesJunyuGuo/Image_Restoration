% !TeX root = ../thuthesis-example.tex

\begin{survey}
\label{cha:survey}

\title{Literature Review}
\maketitle


% \tableofcontents

The field of image restoration has witnessed a paradigm shift with the advent of deep learning technologies. Among these, diffusion models have emerged as particularly potent, owing to their ability to reconstruct high-quality images from corrupted inputs. This review delves into the evolution and contributions of diffusion models, especially focusing on Score-Based Stochastic Differential Equations \cite{score_based_SDE}, Denoising Diffusion Probabilistic Models \cite{DDPM}, and Denoising Diffusion Implicit Models \cite{DDIM}, while also discussing the role of Variational Autoencoders \cite{VAE_diffusion} and the integration of manifold constraint learning \cite{MCG}.  VAEs, as introduced by \cite{vae_model}, marked a significant milestone in image restoration, providing a framework for modeling complex image distributions. Despite their contributions, the quest for models that better capture high-dimensional data distributions led to the exploration of diffusion models, which offer a novel approach to image restoration.       

Diffusion models, particularly DDPM, introduced by \cite{DDPM}, have redefined image restoration strategies. These models employ a reverse diffusion process to generate data from noise, demonstrating unparalleled performance in handling complex image degradations.    


The incorporation of Score-Based SDEs into diffusion models, as explored by \cite{score_based_SDE,song_2}, has further enhanced the field. These models use the data distribution's score-based gradient for a more refined denoising process, leading to superior restoration quality.    

DDIM, a notable advancement over DDPM, was introduced by \cite{DDIM} to improve efficiency in the denoising process. This model significantly accelerates image restoration without compromising output quality, highlighting its potential for real-time applications.   

Manifold constraint learning has been pivotal in ensuring that restored images adhere to natural image characteristics. The integration of this concept with diffusion models, especially in works like \cite{Inverse,PnP,MCG}, demonstrates significant improvements in photorealism and structural coherence.      


Recent literature has expanded the scope of diffusion models in image restoration. The DDRM framework, as discussed by \cite{ddrm}, and the Pseudo Inverse Algorithm \cite{red_diff} offer novel approaches to tackling image restoration tasks, emphasizing the versatility and depth of diffusion-based methodologies. Furthermore, the emergence of consistency models \cite{Consistency} and Stable Diffusion  \cite{vae_model} highlight the ongoing innovation within the domain, promising even more sophisticated solutions to image restoration challenges.       


The exploration of diffusion models in image restoration has ushered in a new era of possibilities, with DDPM, DDIM, and the integration of Score-Based SDEs representing significant advancements. The addition of manifold constraint learning and the exploration of new models like DDRM and the Pseudo Inverse Algorithm further enrich the field. As research continues to evolve, the foundational works and recent innovations collectively point towards a future where image restoration achieves unprecedented levels of accuracy and realism.



\bibliographystyle{unsrtnat}
\bibliography{ref/refs}

\end{survey}
