% !TeX root = ../thuthesis-example.tex

\chapter{补充内容}
\section{插图}


\section{证明}
\begin{lemma}
    Lemma 3. Let $\phi(\cdot)$ be an isotropic multivariate Gaussian density function with mean $\boldsymbol{\mu}$ and variance $\sigma^2 \boldsymbol{I}$. There exists a constant $L$ such that $\forall \boldsymbol{x}, \boldsymbol{y} \in \mathbb{R}^d$,
$$
\|\phi(\boldsymbol{x})-\phi(\boldsymbol{y})\| \leq L\|\boldsymbol{x}-\boldsymbol{y}\|,
$$
where $L=\frac{d}{\sqrt{2 \pi \sigma^2}} e^{-1 / 2 \sigma^2}$.
\end{lemma}
\begin{proof}
    \begin{align}
\|\phi(\boldsymbol{x})-\phi(\boldsymbol{y})\| & \leq \max _{\boldsymbol{z}}\left\|\nabla_{\boldsymbol{z}} \boldsymbol{\phi}(\boldsymbol{z})\right\| \cdot\|\boldsymbol{x}-\boldsymbol{y}\| \\
& =\underbrace{\frac{d}{\sqrt{2 \pi \sigma^2}} \exp \left(-\frac{1}{2 \sigma^2}\right)}_L \cdot\|\boldsymbol{x}-\boldsymbol{y}\|
\end{align}
where the second inequality comes from that each element of $\nabla_{\boldsymbol{z}} \phi(\boldsymbol{z})$ is bounded by $\frac{1}{\sqrt{2 \pi \sigma^2}} \exp \left(-\frac{1}{2 \sigma^2}\right)$.
\end{proof}


\begin{proposition}
     (Jensen gap upper bound (Gao et al., 2017)). Define the absolute cenetered moment as $m_p:=\sqrt[p]{\mathbb{E}\left[|X-\mu|^p\right]}$, and the mean as $\mu=\mathbb{E}[X]$. Assume that for $\alpha>0$, there exists a positive number $K$ such that for any $x \in \mathbb{R},|f(x)-f(\mu)| \leq K|x-\mu|^\alpha$. Then,
\begin{align}
|\mathbb{E}[f(X)-f(\mathbb{E}[X])]| & \leq \int|f(X)-f(\mu)| d p(X) \\
& \leq K \int|x-\mu|^\alpha d p(X) \leq M m_\alpha^\alpha
\end{align}
\end{proposition}

\begin{theorem}[定理1的证明]
    For the given measurement model (6) with $\boldsymbol{n} \sim \mathcal{N}\left(0, \sigma^2 \boldsymbol{I}\right)$, we have
\begin{equation}
  p\left(\boldsymbol{y} \mid \boldsymbol{x}_t\right) \simeq p\left(\boldsymbol{y} \mid \hat{\boldsymbol{x}}_0\right),  
\end{equation}
where the approximation error can be quantified with the Jensen gap, which is upper bounded by
\begin{equation}
    \mathcal{J} \leq \frac{d}{\sqrt{2 \pi \sigma^2}} e^{-1 / 2 \sigma^2}\left\|\nabla_{\boldsymbol{x}} \mathcal{A}(\boldsymbol{x})\right\| m_1,
\end{equation}
where $\left\|\nabla_{\boldsymbol{x}} \mathcal{A}(\boldsymbol{x})\right\|:=\max _{\boldsymbol{x}}\left\|\nabla_{\boldsymbol{x}} \mathcal{A}(\boldsymbol{x})\right\|$ and $m_1:=\int\left\|\boldsymbol{x}_0-\hat{\boldsymbol{x}}_0\right\| p\left(\boldsymbol{x}_0 \mid \boldsymbol{x}_t\right) d \boldsymbol{x}_0$.
\begin{proof}
 \begin{align}
p\left(\boldsymbol{y} \mid \boldsymbol{x}_t\right) & =\int p\left(\boldsymbol{y} \mid \boldsymbol{x}_0\right) p\left(\boldsymbol{x}_0 \mid \boldsymbol{x}_t\right) d \boldsymbol{x}_0 \\
& =\mathbb{E}_{\boldsymbol{x}_0 \sim p\left(\boldsymbol{x}_0 \mid \boldsymbol{x}_t\right)}\left[f\left(\boldsymbol{x}_0\right)\right]
\end{align}

Here, $f(\cdot):=h(\mathcal{A}(\cdot))$ where $\mathcal{A}$ is the forward operator and $h(\boldsymbol{x})$ is the multivariate normal distribution with mean $\boldsymbol{y}$ and the covariance $\sigma^2 \boldsymbol{I}$. Therefore, we have
\begin{align}
J\left(f, p\left(\boldsymbol{x}_0 \mid \boldsymbol{x}_t\right)\right) & =\left|\mathbb{E}\left[f\left(\boldsymbol{x}_0\right)\right]-f\left(\mathbb{E}\left[\boldsymbol{x}_0\right]\right)\right|=\left|\mathbb{E}\left[f\left(\boldsymbol{x}_0\right)\right]-f\left(\hat{\boldsymbol{x}}_0\right)\right| \\
& =\left|\mathbb{E}\left[h\left(\mathcal{A}\left(\boldsymbol{x}_0\right)\right)\right]-h\left(\mathcal{A}\left(\hat{\boldsymbol{x}}_0\right)\right)\right| \\
& \leq \int\left|h\left(\mathcal{A}\left(\boldsymbol{x}_0\right)\right)-h\left(\mathcal{A}\left(\hat{\boldsymbol{x}}_0\right)\right)\right| d P\left(\boldsymbol{x}_0 \mid \boldsymbol{x}_t\right) \\
& \stackrel{\text { (b) }}{\leq} \frac{d}{\sqrt{2 \pi \sigma^2}} e^{-1 / 2 \sigma^2} \int\left\|\mathcal{A}\left(\boldsymbol{x}_0\right)-\mathcal{A}\left(\hat{\boldsymbol{x}}_0\right)\right\| d P\left(\boldsymbol{x}_0 \mid \boldsymbol{x}_t\right) \\
& \stackrel{(c)}{\leq} \frac{d}{\sqrt{2 \pi \sigma^2}} e^{-1 / 2 \sigma^2}\left\|\nabla_{\boldsymbol{x}} \mathcal{A}(\boldsymbol{x})\right\| \int\left\|\boldsymbol{x}_0-\hat{\boldsymbol{x}}_0\right\| d P\left(\boldsymbol{x}_0 \mid \boldsymbol{x}_t\right) \\
& \stackrel{\text { (d) }}{\leq} \frac{d}{\sqrt{2 \pi \sigma^2}} e^{-1 / 2 \sigma^2}\left\|\nabla_{\boldsymbol{x}} \mathcal{A}(\boldsymbol{x})\right\| m_1
\end{align}

where $d P\left(\boldsymbol{x}_0 \mid \boldsymbol{x}_t\right)=p\left(\boldsymbol{x}_0 \mid \boldsymbol{x}_t\right) d \boldsymbol{x}_0$, (b) is the result of Lemma 3, (c) is from the intermediate value theorem, and (d) is from Proposition 2.   
\end{proof}
\end{theorem}




% % 附录中的插图示例(图~\ref{fig:appendix-figure})。

% \begin{figure}
%   \centering
%   \includegraphics[width=0.6\linewidth]{example-image-a.pdf}
%   \caption{附录中的图片示例}
%   \label{fig:appendix-figure}
% \end{figure}


% \section{表格}

% % 附录中的表格示例(表~\ref{tab:appendix-table})。

% \begin{table}
%   \centering
%   \caption{附录中的表格示例}
%   \begin{tabular}{ll}
%     \toprule
%     文件名          & 描述                         \\
%     \midrule
%     thuthesis.dtx   & 模板的源文件,包括文档和注释 \\
%     thuthesis.cls   & 模板文件                     \\
%     thuthesis-*.bst & BibTeX 参考文献表样式文件    \\
%     thuthesis-*.bbx & BibLaTeX 参考文献表样式文件  \\
%     thuthesis-*.cbx & BibLaTeX 引用样式文件        \\
%     \bottomrule
%   \end{tabular}
%   \label{tab:appendix-table}
% \end{table}


% \section{数学表达式}

% % 附录中的数学表达式示例(式\eqref{eq:appendix-equation})。
% \begin{equation}
%   \frac{1}{2 \uppi \symup{i}} \int_\gamma f = \sum_{k=1}^m n(\gamma; a_k) \mathscr{R}(f; a_k)
%   \label{eq:appendix-equation}
% \end{equation}


% \section{参考文献}

% 附录中的参考文献示例(\cite{carlson1981two} 和 \cite{carlson1981two,taylor1983scanning,taylor1981study})。

% \printbibliography
